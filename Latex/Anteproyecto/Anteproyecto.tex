\documentclass[12pt]{article}
\usepackage{graphicx}		    % for insert images
\usepackage[spanish]{babel} 	% for spanish titles
\usepackage[a4paper, left=1in, right=1in, top=1in, bottom=1in]{geometry} 
								% for page size and margin settings
\usepackage{mathtools}          % for greek math symbol formatting
\usepackage{enumitem}           % for control of 'enumerate' numbering
\usepackage{listings}           % for control of 'itemize' spacing
\usepackage{indentfirst}		% package to make first paragraph always indented
\usepackage{hyperref}           % page numbers and '\ref's become clickable
\usepackage{bm}					% for bold maths

% TITLE VARIABLES 

\def\thesistitle{Modelos longitudinales con covariables que varían en el tiempo}
\def\thesisauthorfirst{\textbf{Esteban Cometto}}
\def\thesissupervisorfirst{Maria Del Carmen García}
\def\thesissupervisorsecond{Noelia Castellana}
\def\thesisdate{\today}

%% OTHER USEFUL VARIABLES 

\def\npatients{560}
\def\fullcovname{\emph{adherencia al tratamiento farmacológico}}
\def\covname{\emph{adherencia}}
\def\cvt{covariable que varía en el tiempo}

%% FOR PDF METADATA
\title{\thesistitle}
\author{\thesisauthorfirst\space\thesisauthorsecond}
\date{\thesisdate}

\begin{document}

\begin{titlepage}
    \newcommand{\HRule}{\rule{\linewidth}{0.5mm}}
	\center
	\textsc{\Large Universidad Nacional de Rosario}\\[.7cm]
	\includegraphics[width=25mm]{img/fceye-unr.png}\\[.5cm]
	\textsc{Facultad de Ciencias Económicas y Estadística}\\[0.5cm]
	\textsc{Anteproyecto de Tesina}
	
	\HRule \\[0.4cm]
	{ \huge \bfseries \thesistitle}\\[0.1cm]
	\HRule \\[.5cm]
	
	\begin{minipage}{0.6\textwidth}
	\large
	\emph{Autor:}	\thesisauthorfirst
	\end{minipage}
	\\[.6cm]
	\begin{minipage}{0.6\textwidth}
	\emph{Director:} 	\thesissupervisorfirst \\[.2cm]
	\emph{Codirector:} 	\thesissupervisorsecond
	\end{minipage}
	\\[4cm]
	\vfill
	{\large \thesisdate}\\
	\clearpage
\end{titlepage}

\newpage
\tableofcontents

\newpage
\section{Introducción}

En los estudios longitudinales las unidades experimentales se observan repetidamente en varias ocasiones. Los modelos 
lineales mixtos permiten analizar este tipo de datos, modelando, por un lado, la evolución de la respuesta promedio en 
función del tiempo, mediante efectos fijos, y, por otro lado, la variación entre las respuestas repetidas dentro y entre 
sujetos por medio del error y los efectos aleatorios, respectivamente. En este tipo de estudios es también frecuente contar 
con variables explicativas que se desean incorporar al análisis. Estas variables pueden ser fijas a lo largo de todo el 
período o bien pueden variar a lo largo del seguimiento. La introducción de una \cvt (CVT) al 
modelo produce un importante desafío conceptual. Se consideran algunos aspectos de la interpretación de este tipo de 
covariables, presentando diferentes definiciones y enfoques para incorporarlas en los modelos mixtos.

Un programa de atención y control de pacientes hipertensos iniciado en el año 2014 en Rosario realiza un seguimiento 
exhaustivo de \npatients pacientes. Este programa contempla: efectores no médicos supervisados, tratamiento farmacológico 
genérico para la hipertensión y utilización de un algoritmo terapéutico sistematizado. En cada visita se registran tanto 
características de los pacientes, del tratamiento y de los valores de la tensión arterial. En particular, se desea evaluar 
si la adherencia al tratamiento farmacológico influye en los valores de la tensión arterial sistólica a lo largo del 
seguimiento. Como la variable “\fullcovname” es una CVT estocástica se evaluaran diferentes enfoques para incluirla en un 
modelo longitudinal que pueda explicar el cambio en la tensión arterial sistólica media a lo largo del tiempo. 

\newpage
\section{Objetivos}

\subsection{Objetivo Principal}

Profundizar en el estudio de propuestas metodológicas para utilizar la información obtenida de la \cvt dentro de un 
modelo mixto

\subsection{Objetivos Específicos}

\begin{itemize}
	\item Específicar distintos tipos de CVT
	\item Transformaciones a realizar sobre la CVT antes de incluirla al modelo, incluyendo conversión a covariable fija
	\item Consideraciones sobre interpretación de los parámetros sobre las CVT
	\item Indagar sobre feedback entre la CVT y la variable respuesta
\end{itemize}

\newpage
\section{Metodología}

En un modelo mixto, cada unidad tiene una trayectoria individual caracterizada por parámetros y un subconjunto de esos 
parámetros ahora se consideran aleatorios. La respuesta media es modelada como una combinación de características 
poblacionales que son comunes a todos los individuos (efectos fijos) y efectos específicos de la unidad que son únicos de 
ella (efectos aleatorios).

Se consideran las dos fuentes de variación (intra y entre) presentes en los datos longitudinales. Entonces, este modelo va 
a ser similar al modelo lineal general con respecto a la parte media del mismo, pero se va a diferenciar en cuanto a la 
estructura de covariancia.

El modelo lineal mixto para la unidad i se puede expresar en forma matricial como:
\[ \bm{Y}_i = \bm{X}_i\bm{\beta} + \bm{Z}_i\bm{b}_i + \bm{\varepsilon}_i \quad i = 1, ... n \]
Donde:
\begin{itemize}
	\item $\bm{Y}_i$: Vector de la variable respuesta de la i-ésima unidad, de dimensión $(n_ix1)$
	\item $\bm{X}_i$: Matriz de diseño de la i-ésima unidad, que caracteriza la parte sistematica de la respuesta, 
	de dimensión $(n_ixp)$
	\item $\bm{\beta}$: Vector de parámetros de dimensión $(px1)$
	\item $\bm{Z_i}$: Matriz de diseño de la i-ésima unidad, que caracteriza la parte aleatoria de la respuesta, 
	de dimensión $(n_ixk)$
	\item $\bm{b}_i$: Vector de efectos aleatorios de la i-ésima unidad, de dimensión $(kx1)$
	\item $\bm{\varepsilon}_i$: Vector de errores aleatorios de la i-ésima unidad, de dimensión $(n_ix1)$
	\item $n$: número de individuos
	\item $p$: número de parámetros
\end{itemize}
Además, $\bm{b}_i$ y $\bm{\varepsilon}_i$ son independientes 
\[ \bm{b}_i ~ N_{n_i}(0, R_i) \]
\[ \bm{\varepsilon}_i ~ N_{n_i}(0, D_i) \]
Las matrices $D_i$ y $R_i$ contienen las variancias y covariancias de los elementos de los vectores 
$\bm{b}_i$ y $\bm{\varepsilon}_i$ respectivamente


\end{document}