\documentclass[spanish]{article}
\usepackage[a4paper, left=1in, right=1in, top=1in, bottom=1in]{geometry}
                                % for page size and margin settings
\usepackage{graphicx}		    % for insert images
\usepackage[es-tabla]{babel} 	% for spanish titles
\usepackage[a4paper, left=1in, right=1in, top=1in, bottom=1in]{geometry} 
								% for page size and margin settings
\usepackage{mathtools}          % for greek math symbol formatting
\usepackage{enumitem}           % for control of 'enumerate' numbering
\usepackage{listings}           % for control of 'itemize' spacing
\usepackage{indentfirst}		% package to make first paragraph always indented
\usepackage{hyperref}           % page numbers and '\ref's become clickable
\usepackage{bm}					% for bold maths
\usepackage{setspace}			% for setting interline spacing
\usepackage{amsmath}			% for matrices
\usepackage{tikz} 				% for graphs
\usepackage{multirow}			% for tables
\usepackage{float}				% to manually select placement of tables

\usetikzlibrary{babel}		    % for draw arrows in tikz using babel spanish
\renewcommand{\baselinestretch}{1.5}
\bibliographystyle{ieeetr}
\numberwithin{figure}{subsection}
\numberwithin{equation}{subsection}
\numberwithin{table}{subsection}

% TITLE VARIABLES 

\def\thesistitle{Incorporación de covariables que varían en el tiempo a un modelo mixto}
\def\thesisauthorfirst{\textbf{Esteban Cometto}}
\def\thesissupervisorfirst{Noelia Castellana}
\def\thesissupervisorsecond{Cecilia Rapelli}
\def\thesisdate{\today}

%% OTHER USEFUL VARIABLES 

\def\npatients{560}
\def\fullcovname{adherencia al tratamiento farmacológico}
\def\covname{\textit{adherencia}}
\def\cvt{covariable que varía en el tiempo}
\def\xseqj{$X_{i1}, ..., X_{ij}$}
\def\xseqn{$X_{i1}, ..., X_{in_i}$}
\def\xseqjminus{$X_{i1}, ..., X_{ij-1}$}
\def\yseqj{$Y_{i1}, ..., Y_{ij}$}
\def\yseqn{$Y_{i1}, ..., Y_{in_i}$}
\def\yseqjminus{$Y_{i1}, ..., Y_{ij-1}$}

%% FOR PDF METADATA
\title{\thesistitle}
\author{\thesisauthorfirst\space\thesisauthorsecond}
\date{\thesisdate}

\begin{document}

\begin{titlepage}
    \newcommand{\HRule}{\rule{\linewidth}{0.5mm}}
	\center
	\textsc{\Large Universidad Nacional de Rosario}\\[.7cm]
	\includegraphics[width=25mm]{img/fceye-unr.png}\\[.5cm]
	\textsc{Facultad de Ciencias Económicas y Estadística}\\[0.5cm]
	\textsc{Anteproyecto de Tesina}
	
	\HRule \\[0.4cm]
	{ \huge \bfseries \thesistitle}\\[0.1cm]
	\HRule \\[.5cm]
	
	\begin{minipage}{0.6\textwidth}
	\large
	\textit{Autor:}	\thesisauthorfirst
	\end{minipage}
	\\[.6cm]
	\begin{minipage}{0.6\textwidth}
	\textit{Directora:} 	\thesissupervisorfirst \\[.2cm]
	\textit{Codirectora:} 	\thesissupervisorsecond
	\end{minipage}
	\\[4cm]
	\vfill
	{\large \thesisdate}\\
	\clearpage
\end{titlepage}

\newpage
\tableofcontents

\newpage
\section{Introducción}

%% Tesis mara catalano, silvia camats y redacción mia

Los datos longitudinales están conformados por mediciones repetidas de una
misma variable realizadas a la misma unidad. Estas mediciones surgen de
observar unidades en diferentes ocasiones, es decir en diferentes momentos o
condiciones experimentales.

Dado que las mediciones repetidas son obtenidas de la misma unidad, los datos
longitudinales están agrupados. Las observaciones dentro de un mismo
agrupamiento generalmente están correlacionadas positivamente. Por lo tanto,
los supuestos usuales acerca de la independencia entre las respuestas de cada
unidad y la homogeneidad de variancias frecuentemente no son válidos

El objetivo principal de estos estudios es estudiar los cambios en el tiempo y
los factores que influencian el cambio.

Las ocasiones en las que se registran las mediciones repetidas no
necesariamente serán iguales para todos los individuos, por lo tanto se pueden
obtener tanto estudios balanceados (todos los individuos tienen el mismo número
de mediciones durante un conjunto de ocasiones comunes) como desbalanceados (la
secuencia de tiempos de observaciones no es igual para todos los individuos).
Otra característica de estos datos es que en ocasiones se pueden obtener
valores perdidos, obteniendo datos incompletos aunque se cuente con un estudio
balanceado.

Los modelos mixtos permiten ajustar datos con estas particularidades, donde la
respuesta es modelada por una parte sistemática que está formada por una
combinación de características poblacionales que son compartidas por todas las
unidades (efectos fijos), y una parte aleatoria que está constituida por
efectos específicos de cada unidad (efectos aleatorios) y por el error
aleatorio. Estos modelos permiten, además, hacer predicciones del perfil de una
unidad específica. La selección de la estructura de covariancia apropiada
produce estimadores más eficientes y consecuentemente, pruebas estadísticas más
robustas para los efectos de interés

%% Chapter lalonde

Las covariables en los estudios longitudinales se pueden clasificar en dos
categorias: fijas y variables en el tiempo. Las diferencias entre estos tipos
de covariables pueden llevar a diferentes intereses de investigación,
diferentes tipos de análisis y diferentes conclusiones.

Las covariables fijas son variables independientes que no tienen variación
intra-sujeto, lo que significa que el valor de la covariable no cambia para un
individuo determinado en el estudio longitudinal. Este tipo de covariable se
puede usar para realizar comparaciones entre poblaciones y describir diferentes
tendencias en el tiempo, pero no permite una relación dinámica entre la
covariable y la respuesta.

Las covariables variables en el tiempo (CVT) son variables independientes que
contienen ambas variaciones, intra y entre sujeto, lo que significa que el
valor de la covariable cambia para un individuo determinado a lo largo del
tiempo y además puede cambiar para diferentes sujetos. Una CVT se puede usar
para hacer comparaciones entre poblaciones, describir tendencias en el tiempo y
también la relación dinámica entre la CVT y la respuesta

Se puede ver que las CVT permiten diferentes tipos de relaciones y conclusiones
que las covariables fijas. Por ejemplo, una CVT puede estar involucrada en
efectos acumulados para diferentes valores a través del tiempo (Fitzmaurice y
Lard 1995). Además, ciertas CVT transmiten diferente información que otras. Por
ejemplo, covariables como la edad pueden cambiar a través del tiempo, pero
cambian de manera predecible. Por otro lado, covariables como la precipitación
diaria pueden cambiar a través del tiempo pero no pueden ser predecidas. En
esos casos es importante considerar las relaciones entre la CVT y la respuesta
a través del tiempo.

%% Redacción mia

En el presente informe se cuenta con un programa de atención y control de
pacientes hipertensos iniciado en el año 2014 en Rosario que realiza un
seguimiento exhaustivo de \npatients{} pacientes. Este programa contempla:
efectores no médicos supervisados, tratamiento farmacológico genérico para la
hipertensión y utilización de un algoritmo terapéutico sistematizado. En cada
visita se registra el seguimiento del tratamiento y los valores de la tensión
arterial sistólica. En particular, se desea evaluar si la adherencia al
tratamiento farmacológico influye en los valores de la tensión arterial
sistólica a lo largo del seguimiento. Como la variable “\fullcovname{}” es una
CVT estocástica se evaluaran diferentes enfoques para incluirla en un modelo
longitudinal que pueda explicar el cambio en la tensión arterial sistólica
media a lo largo del tiempo.

Un aspecto a tener en cuenta en este trabajo es que, si bien contamos con mucha
otra información para obtener modelos que describan de mejor manera el
comportamiento de la TAS, nos centraremos en modelos más simples con respecto a
las covariables fijas con el fin de no perder de vista la relación entre la
variable respuesta y la CVT.

\newpage
\section{Objetivos}

\subsection{Objetivo Principal}

Presentar diferentes propuestas metodológicas respecto a la incorporación de
covariables que varían con el tiempo en modelos mixtos para datos
longitudinales.

\subsection{Objetivos Específicos}

\begin{itemize}
	\item Definir los tipos de covariables existentes.
	\item Evaluar propuestas de incorporación de covariables que varían en el
	tiempo en los modelos mixtos.
	\item Aplicar los conceptos vistos en un estudio sobre la tendencia de la
		  presión arterial en el tiempo para pacientes que siguen cierto
		  tratamiento.
\end{itemize}

\newpage
\section{Metodología}

\subsection{Los Datos Longitudinales}

Los datos longitudinales están conformados por mediciones repetidas de una
misma variable realizadas a la misma unidad. Estas mediciones surgen de
observar unidades en diferentes ocasiones, es decir en diferentes momentos o
condiciones experimentales.

Dado que las mediciones repetidas son obtenidas de la misma unidad, los datos
longitudinales están agrupados. Las observaciones dentro de un mismo
agrupamiento generalmente están correlacionadas positivamente.

El objetivo principal de estos estudios es estudiar los cambios en el tiempo y
los factores que influencian el cambio. 

Las ocasiones en las que se registran las mediciones repetidas no
necesariamente serán iguales para todos los individuos, por lo tanto se pueden
obtener tanto estudios balanceados (todos los individuos tienen el mismo número
de mediciones durante un conjunto de ocasiones comunes) como desbalanceados (la
secuencia de tiempos de observaciones no es igual para todos los individuos).
Otra característica de estos datos es que en ocasiones se pueden obtener
valores perdidos, obteniendo datos incompletos aunque se cuente con un estudio
balanceado.

Con el fin de simplificar la notación, se asumirá que los tiempos de medición
son los mismos para todas las unidades y que no hay datos faltantes.

Se obtiene una muestra de $N$ unidades cada una con $n$ mediciones repetidas de
la variable en estudio, observadas en los tiempos $t_1, t_2, ..., t_n$, siendo
entonces el número total de observaciones $N^*=Nn$. Se le llama $Y_{ij}$ a la
medición sobre la unidad $i$ en la ocasión $j$, con $i=1, ..., N; j=1, ..., n$

Asociadas a cada unidad se observan las covariables $X_{ij}$, de las cuales
existen dos tipos: variables en el tiempo (estocásticas) e invariables en el
tiempo (estacionarias)

Existen estudios empíricos que llevan a pensar que existen tres fuentes
potenciales de variabilidad que influyen sobre la correlación entre medidas
repetidas:

\begin{itemize}
	\item \textit{Heterogeneidad entre las unidades:} Refleja la propensión
	natural de las unidades a responder. Los individuos tienen diferentes
	reacciones frente a los mismos estímulos.
	\item \textit{Variación biológica intra-unidad:} Se piensa que la secuencia
	de medidas repetidas de una unidad tiene un comportamiento determinado, que
	produce que las mediciones más cercanas sean más parecidas.
	\item \textit{Error de medición:} Surge debido a los errores de medida, se
	puede disminuir usando instrumentos de medición más precisos
\end{itemize}

Estas tres fuentes de variación pueden clasificarse en \textit{``variabilidad
entre''}, es decir, la variación entre las unidades (heterogeneidad entre
unidades) y \textit{``variabilidad intra''}, es decir, la variación entre las
mediciones de las misma unidad (variación biológica intra-unidad y error de
medición)

Dado que, como se mencionó anteriormente, las mediciones están correlacionadas
entre sí, si se utilizaran las técnicas habituales basadas en la independencia
entre mediciones, los errores estándares nominales van a ser incorrectos, lo
cual nos llevaría a inferencias incorrectas sobre los parámetros del modelo. En
base a esto, surgen técnicas que consideran esa correlación modelando los datos
considerando la modelación de dos estructuras: la parte media y la estructura
de covariancia.

% \subsection{Análisis exploratorio}

% Antes de ajustar algún modelo, lo primero siempre es realizar un análisis
% exploratorio para estudiar cómo se comportan los datos. A continuación se
% presentan técnicas gráficas para cada estructura.

% \textit{Evaluación de la parte media}

% \begin{itemize}
% 	\item \textit{Perfil individual:} Consiste en un gráfico de dispersión en
% 	el cual se representan las respuestas vs las ocasiones. Cada respuesta
% 	tiene un punto y se une con un segmento los puntos de la misma unidad.
% 	Sirven para detectar si hay mucha variabilidad entre y dentro de las
% 	unidades y si hay valores atípicos.
% 	\item \textit{Perfiles promedio por grupo:} En general son más
% 	informativos. Para cada tiempo calculamos un promedio para cada grupo y
% 	luego se unen los puntos. Permiten ver la tendencia de las variables a
% 	través de las ocasiones. Se superponen en un mismo gráfico los perfiles
% 	promedio de cada grupo
% \end{itemize}

% \textit{Evaluación de la estructura de covariancia}

% \begin{itemize}
% 	\item \textit{Matriz de diagrama de dispersión:} Para cada par de ocasiones
% 	se grafican los valores esperados de la respuesta y todos estos gráficos se
% 	acomodan dentro de una matriz. En general se utiliza cuando las ocasiones
% 	son las mismas para todas las unidades.
% 	\item \textit{Gráfico de Draftman:} Es similar al gráfico anterior pero
% 	utilizando variables estandarizadas. La utilización de la variable
% 	respuesta estandarizada ayuda a eliminar la variabilidad de los datos
% 	asociada con diferencias en las medias y variancias en el tiempo,
% 	permitiendo visualizar más claramente el patrón de correlación.
% 	\item \textit{Gráfico PRISM (Partial Regression on Intervenors Scatterplot
% 	Matrix):} Utilizando la variable estandarizada se crea una matriz de
% 	gráficos de dispersión. En la primera diagonal se encuentran gráficos de
% 	dispersión entre la variable respuesta en los tiempos $t_j$ y $t_{j+1}$.
% 	Luego, en la k-ésima diagonal, se obtienen gráficos de regresión parcial de
% 	las respuestas en los tiempos $t_j$ y $t_{j+k}$, ajustadas por las
% 	respuestas en los tiempos intermedios. Estos gráficos permiten ver con
% 	mayor claridad ciertos tipos de estructuras seriales que se dan entre las
% 	medidas repetidas.
% 	\item \textit{Correlograma:} Representa las características que existen
% 	entre las respuestas de los individuos de cada grupo en tiempos que están
% 	separados una cantidad de periodos. Permite analizar cómo evoluciona la
% 	correlación a medida que aumenta el número de rezagos.
% 	\item \textit{Semivariograma:} Cuando los datos están desbalanceados, el
% 	semivariograma permite distinguir las 3 fuentes de variabilidad. Después de
% 	haber estimado un modelo, el mismo permite confirmar si la estructura de
% 	correlación es adecuada. El semivariograma se define como una función:
% 	\[ \gamma(u) = \frac{1}{2} E[(\varepsilon _{ij} - \varepsilon_{ij'})^2] \]
% 	\[ u_{ijj'} = |t_{ij} - t_{ij'}| \]
% 	\[ \widehat{\gamma(u)} = v_{ijj'} = \frac{1}{2} (r_{ij} - r_{ij'}) \] donde
% 	$r_{ij}$ y $r_{ij'}$ son los residuos estandarizados obtenidos después de
% 	ajustar un modelo de regresión considerando las observaciones
% 	independientes.

% 	Se va a obtener un gráfico donde la variabilidad total va a estar dividida
% 	en 3 partes. Si la curva no empieza en cero significa que hay error de
% 	medición, si tiene pendiente quiere decir que hay un error debido a una
% 	causa biológica (correlación serial) y si la misma no llega a la variancia
% 	total significa que se debe explicar la variabilidad entre.
% \end{itemize}

% \subsection{Modelo lineal general para datos longitudinales}

% Si se piensa que existe una tendencia en el tiempo de las respuestas, y esta
% tendencia se puede expresar como una función, se puede escribir o representar a
% las medidas repetidas de una unidad en un vector $Y_i$. Entonces, un modelo
% lineal para representar la evolución en el tiempo va a ser:
% \[ Y_i = X_i\beta + \varepsilon_i; \quad i = 1, ..., N; \quad Y_i = (Y_{i1},
% Y_{i2}, ..., Y_{in_{i}})' \] Donde:

% \begin{itemize}
% 	\item $Y_{ij}$: respuesta obtenida de la i-ésima unidad en la ocasión
% 	$t_{ij}$.
% 	\item $X_i$: matriz de diseño de la i-ésima unidad, de dimensión $(n_i*p)$
% 	\item $\beta $: vector de parámetros de dimensión $(p*1)$
% 	\item $\varepsilon_i$: vector de errores aleatorios de la i-ésima unidad,
% 	de dimensión $(n_i*1)$, este mismo representa todas las fuentes de
% 	variabilidad de los datos longitudinales
% 	\[ \varepsilon_i \sim N_{n_i}(0, \varSigma_i(\theta )) \]
% 	\item $\theta$: vector de parámetros desconocidos de covariancia, de
% 	dimensión $(q*1)$
% \end{itemize}

% \subsection{Modelación de la estructura de covariancia}

% Al tenerse tantos parámetros de variancia $(n)$ y covariancia $n(n-1)/2$ para
% estimar, se proponen modelos específicos para la estructura de correlación. Se
% trata de elegir una estructura que no tenga tantos parámetros. Sin embargo, se
% debe tener cuidado de no seleccionar estructuras demasiado parcas con las que
% se pierda información.

% La matriz de covariancia de cada unidad va a ser función de $\theta$. El número
% de parámetros de este vector depende de la estructura de la matriz.

% A continuación se mencionan algunas estructuras que se pueden utilizar, se
% llamará $R$ a la matriz de correlaciones

% \subsubsection{Arbitraria o no estructurada (datos balanceados)}

% Considera variancias y covariancias distintas entre las mediciones repetidas.
% Siendo $\sigma_{jj'} = Cov(Y_{ij}Y_{ij'})$ se expresa como:
% \[
% \varSigma = 
% \begin{bmatrix}
% 	\sigma_{1}^2 & \sigma_{12}  & \dots  & \sigma_{1n} \\
% 	\sigma_{21}  & \sigma_{2}^2 & \dots  & \sigma_{2n} \\
% 	\vdots 		 & \vdots	    & \ddots & \vdots	   \\
% 	\sigma_{n1}  & \sigma_{n2}  & \dots  & \sigma_{n}^2
% \end{bmatrix}
% \]
% La ausencia de restricciones hace que haya que estimar una gran cantidad de
% parámetros

% \subsubsection{Simetria compuesta (datos balanceados o no balanceados)}

% La correlación entre pares de observaciones es la misma, sin importar la
% cantidad de rezagos entre ellas,
% $Corr(Y_{ij}, Y_{ik}) = \rho $ para todo $j \neq k$
% \[
% R_i =
% \begin{bmatrix}
% 	1      & \rho   & \dots  & \rho \\
% 	\rho   & 1      & \dots  & \rho \\
% 	\vdots & \vdots	& \ddots & \vdots \\
% 	\rho   & \rho   & \dots  & 1
% \end{bmatrix}
% \]

% \subsubsection{Toeplitz (datos equiespaciados)}
% Se plantea para que cualquier par de respuestas que estén igualmente separadas
% en el tiempo la correlación es la misma, $Corr(Y_{ij},Y_{ij+k}) = \rho_{k}$
% para todo $j$ y $k$.
% \[
% R_i =
% \begin{bmatrix}
% 	1      & \rho_1 	& \dots  & \rho_n \\
% 	\rho_1 & 1     		& \dots  & \rho_{n-1} \\
% 	\vdots & \vdots		& \ddots & \vdots \\
% 	\rho_n & \rho_{n-1} & \dots  & 1
% \end{bmatrix}
% \]

% \subsubsection{Autorregresiva de primer orden (datos equiespaciados)}
% Es un caso especial de la estructura anterior, en la que
% $Corr(Y_{ij}, Y_{ij+k}) = \rho^k$. Esta estructura asume que la correlación
% entre medidas repetidas disminuye a medida que aumenta el número de rezagos
% entre ellas.
% \[
% R_i =
% \begin{bmatrix}
% 	1      & \rho 	    & \dots  & \rho^n \\
% 	\rho   & 1     		& \dots  & \rho^{n-1} \\
% 	\vdots & \vdots		& \ddots & \vdots \\
% 	\rho^n & \rho^{n-1} & \dots  & 1
% \end{bmatrix}
% \]

% \subsubsection{Markov (datos no equiespaciados)}

% Es una generalización de la estructura autorregresiva para mediciones no
% equiespaciadas. $Corr(Y_{ij}, Y_{ij'}) = \rho^{d_{jj'}}$, donde
% $d_{jj'} = |t_{ij} - t_{ij'}|$ para todo $j \neq j'$.
% \[
% R_i =
% \begin{bmatrix}
% 	1      		  & \rho^{d_{12}} & \dots  & \rho^{d_{1n}} \\
% 	\rho^{d_{21}} & 1     		  & \dots  & \rho^{n-1} \\
% 	\vdots 		  & \vdots		  & \ddots & \vdots \\
% 	\rho^n 		  & \rho^{n-1}    & \dots  & 1
% \end{bmatrix}
% \]

\subsection{Modelos lineales mixtos}

En estos modelos, cada unidad tiene una trayectoria individual caracterizada
por parámetros y un subconjunto de esos parámetros ahora se consideran
aleatorios. La respuesta media es modelada como una combinación de
características poblacionales que son comunes a todos los individuos (efectos
fijos) y efectos específicos de la unidad que son únicos de ella (efectos
aleatorios).

Se consideran las dos fuentes de variación (intra y entre) presentes en los
datos longitudinales. Entonces, este modelo va a ser similar al modelo lineal
general con respecto a la parte media del mismo, pero se va a diferenciar en
cuanto a la estructura de covariancia.

El modelo lineal mixto para la unidad $i$ se puede expresar en forma matricial
como:

\[ Y_i = X_i\beta + Z_ib_i + \varepsilon_i; \quad i = 1, ..., N;
\quad Y_i = (Y_{i1}, Y_{i2}, ... Y_{in_{i}})' \]

Donde:

\begin{itemize}
	\item $Y_i$: Vector de la variable respuesta de la i-ésima unidad, de
	dimensión $(n_i*1)$
	\item $X_i$: Matriz de diseño de la i-ésima unidad, que caracteriza la
	parte sistemática de la respuesta, de dimensión $(n_i*p)$
	\item $\beta$: Vector de parámetros de dimensión $(p*1)$
	\item $Z_i$: Matriz de diseño de la i-ésima unidad, que caracteriza la
	parte aleatoria de la respuesta, de dimensión $(n_i*k)$
	\item $b_i$: Vector de efectos aleatorios de la i-ésima unidad, de
	dimensión $(k*1)$
	\item $\varepsilon_i$: Vector de errores aleatorios de la i-ésima unidad,
	de dimensión $(n_i*1)$
\end{itemize}

$\varepsilon_i$ y $b_i$ son independientes.

\[ \varepsilon_i \sim N_{n_i}(0, R_i) \]
\[ b_i \sim N_k(0, D_i) \]

Las matrices $D_i$ y $R_i$ contienen las variancias y covariancias de los
elementos de los vectores $b_i$ y $\varepsilon_i$ respectivamente. A partir de
este modelo se obtiene:

\begin{itemize}
	\item $E(y_i/b_i) = X_i\beta + Z_ib_i$ (media condicional o específica de
	la i-ésima unidad)
	\item $E(Y_i) = X_i\beta$ (media marginal)
	\item $Cov(Y_i/b_i) = R_i$ (variancia condicional)
	\item $Cov(Y_i) = Z_iD_iZ'_i + R_i = \varSigma_i$ (variancia marginal)
\end{itemize}

Generalmente, la matriz $D_i$ adopta una estructura de covariancia arbitraria,
mientras que la matriz $R_i$ adopta cualquiera de las vistas anteriormente

\subsection{Estimación de los parámetros del modelo}

Bajo el supuesto de que $\varepsilon_i$ y $b_i$ se distribuyen normalmente se
pueden usar métodos de estimación basados en la teoría de máxima verosimilitud,
cuya idea es asignar a los parámetros el valor más probable en base a los datos
que fueron observados. Se usarán para estimar los parámetros de la parte media
y los de las estructuras de covariancia los métodos de máxima verosimilitud
(ML) y máxima verosimilitud restringida (REML) respectivamente

\subsubsection{Método de máxima verosimilitud (ML)}

Bajo el supuesto de que $Y_i \sim N_{n_i}(X_i\beta, \varSigma_i)$ y las $Y_i$
son independientes entre sí, se obtiene la siguiente función de
log-verosimilitud:

\begin{equation}
\label{ML}
	l = -\frac{1}{2} \sum_{i=1}^{N}n_i ln(2\pi) - \frac{1}{2}ln|\varSigma_i| -
	\frac{1}{2} \sum_{i=1}^{N} [(Y_i - X_i\beta)'
	\varSigma_i^{-1} (Y_i - X_i\beta)]
\end{equation}

Siendo $\varSigma_i$ función del vector $\theta$ que contiene los parámetros de
covariancia.

La ecuación anterior se debe derivar con respecto a $\beta$ y $\theta$ y luego
debe igualarse a cero, de esta manera se obtienen sus estimadores. Cuando
$\theta$ es desconocido (lo que generalmente sucede) se obtiene una ecuación no
lineal, por lo que no se puede obtener una expresión explícita de
$\hat{\theta}$, para encontrar su solución se recurren a algoritmos numéricos.
El estimador del vector $\beta$ resulta:

\[ \hat{\beta} = (\sum_{i=1}^{N} X_i'\hat{\varSigma_i}^{-1}X_i)^{-1}
\sum_{i=1}^{N} X'_i\hat{\varSigma_i}^{-1}Y_i \]

El estimador $\hat{\beta}$ resulta insesgado de $\beta$. Cuando $\theta$ es
conocido se conoce la distribución exacta del estimador. Sin embargo, cuando es
desconocido, no se puede calcular de manera exacta la matriz de covariancias de
$\hat{\beta}$. Si el número de unidades es grande se puede demostrar que
asintóticamente:

\[ \hat{\beta} \sim N_p(\beta, V_{\beta}) \quad donde \quad V_{\beta} =
(\sum_{i=1}^{N} X'_i\hat{\varSigma_i}^{-1}X_i)^{-1} \]

% AGREGAR EL PROBLEMA CUANDO EL SUPUESTO NO SE CUMPLE

\subsubsection{Método de máxima verosimilitud restringida (REML)}

El inconveniente que posee el método de MV es que los parámetros de covariancia
resultan sesgados. Es decir, a pesar de que la estimación de $\beta$ resulta
insesgada, no pasa lo mismo con $\theta$. Si el tamaño de muestra es chico, los
parámetros que representan las variancias van a ser demasiado pequeños, dando
así una visión muy optimista de la variabilidad de las mediciones, es decir, se
subestiman los parámetros de covariancia. El sesgo se debe a que en la
estimación MV no se tiene en cuenta que $\beta$ es estimado a partir de los
datos.

El método REML separa la parte de los datos usada para estimar $\beta$ de
aquella usada para estimar los parámetros de $\varSigma_i$, la función de
log-verosimilitud restringida que se propone es:

\begin{equation}
\label{REML}
	l^* = -\frac{1}{2} \sum_{i=1}^{N}n_i ln(2\pi) - \frac{1}{2}ln|\varSigma_i| -
	\frac{1}{2} \sum_{i=1}^{N} [(Y_i - X_i\beta)'
	\varSigma_i^{-1} (Y_i - X_i\beta)] -
	- \frac{1}{2} ln |\sum_{i=1}^{N} X'_i \hat{\varSigma_i^{-1} X_i}|
\end{equation}

Maximizando esta funcion con respecto a $\beta$ y $\theta$ se obtiene:

\[ \hat{\beta} = (\sum_{i=1}^{N} X'_i \hat{\varSigma}_i^{-1} X_i)^{-1}
\sum_{i=1}^{N} X'_i \hat{\varSigma}_i^{-1} Y_i\]

Donde $\hat{\varSigma}_i$ es el estimador REML de ${\varSigma_i}$

% \subsection{Pruebas de hipótesis}

% Generalmente, la inferencia en estos modelos se centra en los parámetros de la
% parte media. Es decir, en combinaciones lineales de los parámetros de $\beta$.
% La hipótesis lineal general para los test se plantea construyendo dichas
% combinaciones a través de $L'\beta$, siendo $L'$ una matriz de dimensión
% $(r*p)$.

% El estimador de $L'\beta$ resulta $L'\hat{\beta}$ y asintóticamente su
% distribución muestral es aproximadamente:

% \[ L'\beta \sim N_p(L'\beta, L'V_{\beta}L) \]

% Para probar las hipótesis se proponen dos métodos basados en la función de
% verosimilitud:

% \subsubsection{Test de Wald}

% Se plantean las hipótesis:

% \[ H_0) L'\beta = 0 \quad H_1) L'\beta \neq 0 \]

% La estadística de prueba resulta:

% \[ W = (L'\hat{\beta})' (L'\hat{V}_{\beta}L)^{-1} (L'\hat{\beta}) \]

% Donde $W$ se distribuye aproximadamente como $\chi_r^2$.

% Este test provee inferencias válidas cuando el $N$ es grande, ya que utiliza la
% aproximación asintótica a la distribución Normal. Si el $N$ es chico se propone
% reemplazar la distribución de la chi-cuadrado por una $F$ de Snedecor. El
% problema con el uso de esta estadística es que no se conocen los grados de
% libertad del denominador y deben ser calculados con los datos. 

% %Algunos métodos para estimar los grados de libertad son: Contaiment,
% %Within-Between, Satterthwaite y Kenward y Roger, entre otros
% %(\textit{Fitzmaurice et al., 2004})

% \subsubsection{Test de cociente de verosimilitud}

% Se basa en la teoría asintótica y va a suplir las dificultades del test de
% Wald. El test se obtiene comparando las verosimilitudes de dos modelos, uno de
% los cuales incorpora la restricción $L'\beta = 0$ (modelo reducido) y el otro
% no tiene restricciones (mdoelo completo). Estos dos modelos están anidados, ya
% que el modelo reducido es un caso particular del modelo completo.

% Al igual que anteriormente, se plantean las hipótesis:

% \[ H_0) L'\beta = 0 \quad H_1) L'\beta \neq 0 \]

% La función de log-verosimilitud maximizada del modelo completo es:

% \[ -\frac{1}{2} \sum_{i=1}^{N}n_i ln(2\pi) - \frac{1}{2}ln|\varSigma_i| -
% \frac{1}{2} \sum_{i=1}^{N} [(Y_i -
% X_i\hat{\beta})' \varSigma_i^{-1} (Y_i - X_i\hat{\beta})] \]

% Y la del modelo reducido es:

% \[ -\frac{1}{2} \sum_{i=1}^{N}n_i ln(2\pi) - \frac{1}{2}ln|\varSigma_i| -
% \frac{1}{2} \sum_{i=1}^{N} [(Y_i - X_i\beta_0)' \varSigma_i^{-1}
% (Y_i - X_i\beta_0)] \]

% Donde $\beta_0$ es el vector de parámetros resultante de haber impuesto
% $L'\beta = 0$.

% Si el $N$ es grande, la distribución muestral aproximada de la estadística es:

% \[ G^2 = -2 (\hat{l}_{red} - \hat{l}_{comp}) \sim \chi_v^2 \]

% Donde $v$ es el número de parámetros del modelo completo menos el número de
% parámetros del modelo reducido

% Si lo que se quiere comparar mediante este test es la estructura media de dos
% modelos, se recomienda estimar los parámetros mediante el método ML. En cambio,
% si lo que se desea comparar son patrones de covariancia anidados entre dos
% modelos se recomienda estimar los parámetros con el método REML.

% Cuando se desea hacer inferencia sobre las componentes de variancia (por
% ejemplo, postular que una variancia es nula), la diferencia entre los dos
% modelos va a ser de $k+1$ parámetros (1 de variancia y $k$ de covariancia). Si
% sucede esto, la distribución de la estadística $G^2$ ya no será una
% chi-cuadrado común, si no que será una mezcla entre dos chi-cuadrado, una de
% $k$ grados de libertad y otra de $k+1$ grados de libertad.

% % Si bien hay tablas para esta estadística, se recomienda en general utilizar
% % en vez de un nivel de significación del 5%, uno del 10%, para no tener
% % problemas de elegir un modelo demasiado simple

% \subsection{Elección entre modelos de covariancia}

% Cuando los modelos no están anidados, como sucede generalmente cuando se
% plantean modelos con distintas estructuras de covariancia, no se puede usar el
% método del cociente de verosimilitud para compararlos.

% Como en la matriz de covariancia siempre intervienen los residuos, y en ellos
% aparecen los parámetros de la parte media del modelo, para asegurarse que la
% parte media esté bien especificada se elige un modelo maximal (modelo con todos
% los parámetros que queremos incorporar). Dado dicho modelo, se plantean
% distintos modelos que se van a diferenciar únicamente en la estructura de la
% matriz de covariancia.

% Los enfoques que se proponen a continuación se basan en la comparación de
% versiones penalizadas de las log-verosimilitudes de los modelos. Como es
% conocido, a medida que se incorporan parámetros a los modelos, mayor va a ser
% la verosimilitud. Para comparar modelos con distinto número de parámetros, se
% penalizan a los mismos, surgiendo varios criterios A continuación se destacan
% dos de estos:

% \begin{itemize}
% 	\item \textit{Criterio de Akaike (AIC)}: $-2\hat{l} + 2p$
% 	\item \textit{Criterio bayesiano se Schwarz (BIC)}: $-2\hat{l} + p ln(N)$
% \end{itemize}

% Donde:

% $\hat{l}$: log-verosimilitud maximizada del modelo

% $p$: número de parámetros del modelo

% El $BIC$ penaliza más la verosimilitud dando modelos más sencillos. El criterio
% para seleccionar un modelo es aquel que minimice los valores de $AIC$ o $BIC$


% \subsection{Predicción de los efectos aleatorios}

% En muchas aplicaciones la inferencia está centrada solamente en los efectos
% fijos. Estos parámetros se interpretan como los cambios de la respuesta media
% en el tiempo, pero muchas veces se desea conocer además los perfiles
% individuales, por lo que necesitamos conocer los valores de $b_i$. Como $b_i$
% es una cantidad aleatoria se habla de predicción de $b_i$.

% Llamando $c(Y_i)$ al predictor de los efectos aleatorios, se debe minimizar
% $E[b_i - c(Y_i)]^2$. La función $c(Y_i)$ que hace que esa esperanza sea lo más
% chica posible se llama esperanza condicional de $b_i$ dado $Y_i$ y se
% simboliza:

% \[ E(b_i/Y_i) = D_iZ'_i\varSigma_i^{-1} (Y_i - X_i\beta) \]

% Esto se conoce como mejor predicción lineal insesgada (BLUP) y el BLUP empírico
% (EBLUP) o estimador empírico de Bayes resulta:

% \[ \hat{b_i} = \hat{D_i}Z'_i\hat{\varSigma_i}^{-1} (Y_i - X_i\hat{\beta}) \]

% A través de esto se pueden conocer los perfiles individuales, siendo
% $\hat{Y_i} = X_i\hat{\beta} + Z_i\hat{b_i}$,
% operando algebraicamente se llega a la expresión:

% \[ \hat{Y_i} = (\hat{R_i}\hat{\varSigma_i}^{-1}) X_i\beta + (I_n -
% \hat{R_i}\hat{\varSigma_i}^{-1}) Y_i \]

% Resulta que el perfil individual estimado es un promedio ponderado entre el
% perfil de la respuesta media poblacional ($X_i\hat{\beta}$) y el perfil de la
% respuesta observada en la unidad $i$ $(Y_i)$.

% \subsection{Examen de residuos}

% Como en todo análisis de datos en donde se utiliza un modelo estadístico, es
% necesario realizar un estudio del cumplimiento de los supuestos del modelo
% utilizando los gráficos de residuos.

% El modelo lineal mixto para la unidad $i$ es:

% \[ Y_i = X_i\beta + Z_i\beta + \varepsilon_i \]

% Con $b_i$ y $\varepsilon_i$ independientes y
% $\varepsilon \sim N_{n_i}(0, R_i)$, $b_i \sim N_k(0, D_i)$.

% En los modelos lineales mixtos se va a distinguir entre tres tipos de residuos:

% \subsubsection{Residuos marginales}

% Se definen como la diferencia entre el vector de respuestas y su media marginal
% estimada.

% \[ r_{iM} = Y_i - \widehat{E(Y_i)} = Y_i - X_i\hat{\beta} \]

% \subsubsection{Residuos condicionales}

% Se definen como la diferencia entre el vector de respuestas y su media
% condicional estimada.

% \[ r_{iC} = Y_i - \widehat{E(Y_i/b_i)} = Y_i - X_i\hat{\beta} - Z_i\hat{b_i} \]

% \subsubsection{Residuos de los efectos aleatorios}

% \[ r_{iEA} = \widehat{E(Y_i/b_i)} - \widehat{E(Y_i)} = Z_i\hat{b_i} \]

% Estos residuos así calculados reciben el nombre de residuos ordinarios, van a
% estar correlacionados y sus variancias no van a ser necesariamente contrastes.
% Para evitar este inconveniente es necesario realizar una estandarización de los
% mismos:

% \begin{itemize}
% 	\item Si se divide el residuo por $\sqrt{\widehat{Var(r_i)}}$ se obtienen
% 	los residuos estudentizados.
% 	\item Si se divide el residuo por $\sqrt{\widehat{Var(Y_i)}}$ se obtienen
% 	los residuos de Pearson.
% \end{itemize}

% Se define como residuo puro al residuo que depende de solo los componentes
% fijos y del error que predice, y residuo confundido si además de depender de
% las componentes fijas y del error que predicen están en función de otros
% residuos.

% En el word hay una tabla que explica como usar cada residuo

%%%%%%%%%%%%%%%%%%%%%%%%%%%%%%%%%%%%%%%%%%%%%%%%%%
% DIGGLE
%%%%%%%%%%%%%%%%%%%%%%%%%%%%%%%%%%%%%%%%%%%%%%%%%%

\newpage
\section{Covariables que varían en el tiempo}

En los estudios longitudinales, las variables independientes pueden ser
clasificadas en dos categorías: covariables fijas en el tiempo, es decir que no
varían en el tiempo (CNVT) o covariables que varían en el tiempo (CVT). La
diferencia entre ellas puede conducir a diferentes enfoques de análisis así
como también a diferentes conclusiones.

Las CNVT son variables independientes que no presentan variación intra-sujeto,
es decir, los valores de estas covariables no cambian a lo largo del estudio
para un individuo en particular. Por ejemplo, el sexo biológico de una persona
o el grupo de tratamiento.

Las CVT son variables independientes que incluyen tanto la variación
intra-sujeto y la variación entre-sujetos. Esto significa que, para un
individuo en particular, el valor de la covariable cambia a través del tiempo y
puede cambiar también entre diferentes individuos. Por ejemplo, valor de la
presión arterial o condición de fumar (si/no).

Tanto las CNVT y las CVT ueden ser utilizadas para realizar comparaciones entre
poblaciones y describir diferentes tendencias a lo largo del tiempo. Sin
embargo, sólo las CVT permiten describir una relación dinámica entre la
covariable y la variable respuesta.

\subsection{Covariables estocásticas y no estocásticas}

Las CVT no estocásticas son covariables que varían sistemáticamente a través
del tiempo pero son fijas por diseño del estudio o bien su valor puede
predecirse. En cambio, las CVT estocásticas son covariables que varían
aleatoriamente a través del tiempo, es decir, los valores en cualquier ocasión
no pueden ser estimados ya que son gobernados por un mecanismo aleatorio.
Ejemplos de las primeras son: tiempo desde la visita basal, edad, grupo de
tratamiento en los estudios cross-over. Ejemplos de las segunas son: valor del
colesterol, ingesta de alcohol (si/no), ingesta de grasas, etc.

\subsection{Covariables exógenas y endógenas}

Se dice que una CVT es exógena cuando los valores actuales y anteriores de la
respuesta en la ocasión $j (Y_{i1}, ..., Y_{ij})$, dados los valores actuales y
precedentes de la CVT $(X_{i1}, ..., X_{ij})$, no predicen el valor posterior
de $X_{ij+1}$. Más formalmente, una CVT es exógena cuando:

\begin{equation}
	\label{exogeneidad}
	f(X_{ij+1}|X_{i1}, ..., X_{ij}, Y_{i1}, ..., Y_{ij}) =
	f(X_{ij+1}|X_{i1}, ..., X_{ij})
\end{equation}

Y en consecuencia:

\begin{equation}
	\label{exogeneidad debil}
	E(Y_i|X_i) = E(Y_i|X_{i1}, ..., X_{in_i}) = E(Y_i|X_{i1}, ..., X_{ij})
\end{equation}

Esta definición implica que la respuesta en cualquier momento puede depender de
valores previos de la variable respuesta y de la CVT, pero será independiente
de todos los demás valores de la covariable. Por ejemplo, en un estudio
longitudinal en donde se evalúa si el nivel de polución en el aire está
asociado a la función pulmonar, es de esperar que el nivel de polución del aire
en una determinada ocasión dependa de los niveles observados previamente, pero
no se espera que dependa de los niveles de la función pulmonar observados
previamente en el sujeto.

Una CVT que no es exógena se define como endógena. Por ejemplo, cuando se
evalúa si la cantidad de actividad física está asociada al nivel de glicemia.
El nivel de actividad física en un determinado momento puede estar (o no)
asociado a niveles previos y también puede estar asociado a valores previos de
la glicemia (un paciente con valor de glicemia alto en una visita puede decidir
aumentar su nivel de actividad física para reducir este valor)

Es posible examinar empíricamente la suposición de que una CVT es exógena al
considerar modelos de regresión para la dependencia de $X_{ij}$ en
$Y_{i1}, ... Y_{ij-1}$ (o en alguna función conocida de
$Y_{i1}, ... Y_{ij-1}$) y $X_{i1}, ..., X_{ij-1}$ (o en alguna función conocida
de $X_{i1}, ... X_{ij-1}$). La ausencia de cualquier relación entre $X_ij$ y
$Y_{i1}, ... Y_{ij-1}$, dado el perfil de la covariable anterior
$X_{i1}, ..., X_{ij-1}$, proporciona soporte para la validez de la suposición
de que la CVT es exógena.

A los parámetros de regresión se les puede dar una interpretación causal sólo
cuando se puede asumir que las CVT son exógenas con respecto a la variable
respuesta.

\subsection{Otros tipos de CVT}

Trabajos más recientes han definido una nueva categorización de las CVT para
facilitar las interpretaciones y los métodos de estimación adecuados para el
modelo. Se pueden definir cuatro tipos de CVT relacionados con el grado de no
exogeneidad con respecto a la respuesta.

\subsubsection{CVT tipo I}

Se clasifica una CVT como de tipo I si satisface:

\begin{equation}
	\label{CVT tipo I}
	E(Y_{ij}|X_{i1}, ..., X_{in_i}) = E(Y_{ij}|X_{ij})
\end{equation}

En otras palabras, una CVT se considera de tipo I si la variable respuesta en
la j-ésima ocasión es independiente de todos los valores de la CVT en
diferentes momentos, aún de los previos a la ocasión. Variables que involucran
cambios predecibles en el tiempo son tratadas como CVT tipo I, por ejemplo la
edad o el momento de observación.

\subsubsection{CVT tipo II}

Una CVT se clasficia de tipo II si:

\begin{equation}
	\label{CVT tipo II}
	E(Y_{ij}|X_{i1}, ..., X_{in_i}) = E(Y_{ij}|X_{i1}, ..., X_{ij})
\end{equation}

Cabe destacar que la clase de covariables de tipo I es un subconjunto de la
clase de covariables de tipo II. Esta condición dice que el proceso de la CVT
$X_{ij+1}, ..., X_{in_i}$ no se ve afectado por la respuesta $Y_{ij}$.

En otras palabras, la variable respuesta en la j-ésima ocasión puede estar
asociada a valores previos de la CVT. Esta definición es similar pero no
equivalente a la definición de exogeneidad. Se puede demostrar que la
exogeneidad es suficiente es condición suficiente para que una CVT sea de tipo
II. Este tipo de CVT incluyen covariables que pueden tener una asociación
rezagada con la respuesta (los valores anteriores de la CVT pueden afectar a la
respuesta actual) pero los valores de la covariable en un momento determinado
no se verán afectados por los valores previos de la variable respuesta. Un
ejemplo de este tipo de CVT es el ``tratamiento farmacológico para la
hipertensión arterial'' con la variable respuesta ``presión arterial''.

\subsubsection{CVT tipo III}

Se clasifica a una CVT como de tipo III si no es de tipo II. No se asume
independencia entre la respuesta y la covariable, por lo tanto, uede existir un
\textit{feedback} entre ambas en donde los valores de la CVT pueden estar
afectados por valores previos de la variable respuesta. Un ejemplo de este tipo
de CVT es el ``tratamiento farmacológico para la hipertensión arterial'' con la
variable respuesta ``infarto de miocardio''. Mientras que es esperable que la
medicación impacte en la probabilidad de tener un infarto de miocardio, también
tener un infarto de miocardio puede impactar en el cambio del tratamiento
farmacológico.

\subsubsection{CVT tipo IV}

Una CVT se define como de tipo IV si:

\begin{equation}
	\label{CVT tipo IV}
	E(Y_{ij}|X_{i1}, ..., X_{in_i}) = E(Y_{ij}|X_{ij+1}, ..., X_{in_i})
\end{equation}

La CVT puede estar asociada con valores previos de la variable respuesta, pero
la variable respuesta no está asociada con valores previos de la covariable,
está sólo asociada con el valor observado de la covariable en la misma ocasión.
Un ejemplo de este tipo de CVT es ``presión arterial'' con la variable
respuesta ``peso''. En una determinada ocasión hay relación entre ambas
variables, es esperable que valores previos del peso impacten en la presión
arterial pero no al revés

\subsection{Covariables rezagadas}

En la mayoria de los casos, se suele utilizar solo la exposición que ocurre
antes del tiempo $t$ para predecir $Y_{it}$. Sin embargo, en algunas
aplicaciones, el historial completo de la covariable \xseqn{} está disponible y
es considerado como potencial predictor de la respuesta. En otras, solo un
pequeño subconjunto de las mediciones más recientes son usados, ya que se
supone que el efecto en la respuesta está concentrado en ellas. En cualquier
caso, el uso de más de una covariable rezagada puede llevar a predictores
altamente correlacionados, lo que lleva a preguntarse sobre la elección de
cuantos predictores rezagados utiliar y sobre la estructura de sus
coeficientes.

\subsubsection{Una sola covariable rezagada}

En algunas aplicaciones hay justificacion previa para considerar la covariable
en un solo rezago $k$ momentos antes de la medición de la respuesta. Por
ejemplo, muchos agentes farmacológicos son rápidamente limpiados del cuerpo,
por lo que sólo mantienen efectos por una corta duración. En este caso, si la
covariable es exógena, puede ajustarse el modelo mixto sin más consideraciones.
Lo más común es que se desconozca el valor $k$ apropiado y se consideren varias
opciones diferentes.

\subsubsection{Múltiples covaribales rezagadas}

La literatura de series de tiempo ha considerado modelos tanto para infinitos o
finitos rezagos de la covariable. Dado que los datos longitudinales son
tipicamente series de tiempo cortas, se puede proponer un modelo de menor
dimensión para los coeficientes de las covariables rezagadas. En los modelos
rezagados distribuidos, los coeficientes rezagados se asume que siguen una
función paramétrica suave de orden inferior. Por ejemplo, para un rezago finito
$L$, se puede usar un modelo polinomial de orden $p$, con $p < L$ para obtener
coeficientes de regresión suaves.

\[ Y_{it} = \beta_0 + \beta_1 X_{it-1} + \beta_2 X_{it-2} + ... +
\beta_L X_{it-L}, \]
\[ \beta_j = \gamma_0 + \gamma_1 j + \gamma_2 j^2 + ... + \gamma_p j^p \]

A pesar de que los modelos rezagados distribuidos permiten modelar
parsimoniosamente las covariables rezagadas múltiples, la especificación de del
número de rezagos, $L$, y el órden del modelo para el coeficiente, $p$, deben
ser consideradas. Ésto puede realizarse a través de tests para modelos
anidados, como el test del cociente de verosimilitud o el test de Wald.

\subsection{Funcion de las covariables rezagadas}

Una alternativa cuando se quiere utilizar la información de las covariables
rezagadas pero quiere mantenerse un modelo parsimonioso, es decir, con el menor
número posible de variables independientes, es resumir a través de una función
la información de éstas en una sola covariable. Un ejemplo puede ser el valor
promedio o acumulado hasta la ocasión actual. Sin embargo, la elección de está
funcion dependerá del tipo de problema a analizar. Cabe destacar que, al igual
que con toda medida resumen, al usar este tipo de covariables se pierde
parte de la información.

\section{Formas de introducir una CVT al modelo}

\subsection{Convertirla en CNVT}

Una solución rápida al problema de las CVT es transformarla en una CNVT, esto
se puede lograr resumiendo la información de la misma mediante alguna función
como el promedio de los valores de cada individuo y dejarlo fijo a través del
tiempo. También podria usarse su valor máximo, mínimo o cualquier
transformación que resulte de interés en el estudio. El problema de este
enfoque es que se pierde mucha información, dado que se usa una covariable más
simple que no refleja la relación dinámica entre la covariable y la respuesta
en el tiempo.

\subsection{CVT exógena}

Si la CVT es exógena se puede introducirse al modelo en su forma original o con
cualquiera de sus transformaciones mencionadas en la sección anterios, dado que
no habrá problemas con la estimación de los parámetros y pueden recibir una
interpretación causal

\subsubsection{Dividir su efecto en efecto entre-persona e intra-persona}

Cuando se consideran CVT, el modelo lineal mixto puede ser ajustado considerando
dos componentes que reflejen tanto la variación intra-persona y la variación
entre-persona respecto a la CVT. Por lo tanto, el término del modelo que
representa a la covariable se puede descomponer en dos términos:

\[
	\beta X_{ij} \rightarrow \beta_W (X_{ij} - \overline{X}_i)
	+ \beta_B \overline{X}_i
\]

Donde, $\overline{X}_i$ representa el promedio de todos los valores observados
en el tiempo de la CVT para el individuo $i$, $\beta_W$ representa el cambio
esperado en la media de la variable respuesta asociado con cambios de la CVT
dentro de las persona y $\beta_B$ representa el cambio esperado en la media de
la variable respuesta asociado con cambios de la CVT entre personas

\paragraph{Consideraciones con CVT dicotómicas} \mbox{} \\

Cuando tenemos una CVT dicotómica, 0 indica la ausencia de la variable y 1
indica la presencia, entonces $\overline{X}_i$ la proporción en la que una
persona presentó dicha covariable, por lo que el método anterior resultará en
valores extraños para $\beta_W$. Por ejemplo, si la CVT se presentó en el 50\%
de las ocasiones, $\overline{X}_i$ tendrá un valor de 0.5 y entonces el término
que acompaña a $\beta_W$ será de -0.5 en las ocasiones que la CVT dicotómica no
se presente y 0.5 en las que se presente. En términos de la estimación del
modelo esto no genera ningún problema, pero será raro en la interpretación de
los parámetros, dado que el parámetro $\beta_W$ estará siempre presente (nunca
estará acompañado de un 0). Una forma de evitar esto es diviviendo el efecto
entre-persona e intra-persona de la siguiente manera:

\[
	\beta X_{ij} \rightarrow \beta_W X_{ij} + \beta_B \overline{X}_i
\]

\subsection{CVT endógena}

Frente a este escenario, Pepe y Anderson (1994) recomendaron plantear el modelo
longitudinal marginal y realizar las estimaciones mediante GEE (ecuaciones de
estimación generalizadas) con estructura de correlación independiente ya que
este es siempre consistente, por lo que es una opción ``segura''. La estructura
de correlación independiente generalmente tiene una alta eficiencia para la
estimación de los coeficientes asociados a CNVT. Sin embargo, para las CVT,
Fitzmaurice (1995) muestra que esta estructura puede resultar en una pérdida
sustancial de eficiencia para la estimación de los coeficientes asociados a las
CVT y proporciona un ejemplo en el que es sólo un 60\% eficiente en relación con
la estructura de correlación verdadera.

Por otro lado, Lai y Small (2007) propusieron utilizar el ``Método generalizado
de los momento'' (GMM) (Hansen, 1982). En este método de estimación se puede
incorporar información sobre la naturaleza de la CVT que se está analizando. Lai
y Small (2007) definieron 3 tipos de CVT y luego Lalonde et. al (2014)
definieron un cuarto tipo de CVT. 

\newpage

\section{Aplicación}

Se cuenta con un programa de atención y control de pacientes hipertensos
iniciado en el año 2014 en Rosario que realiza un seguimiento exhaustivo de
\npatients{} pacientes de entre 30 y 86 años (M=58.84, DS=9.87) y, de los
cuales un 49.28\% son hombres. Las visitas se realizaron una vez por mes
durante 7 meses desde la primera consulta y en cada una de ellas se registró si
la persona estaba adhiriendo correctamente al tratamiento y el valor de la
tensión arterial sistólica (TAS) (M=134.10, DS=16.07).

A fines de centrarse en la CVT se dejaron de lado algunas covariables fijas
derivadas de estudios de laboratorio, manteniendo solo algunas covariables
sociodemográficas de interés para lograr una estabilidad entre modelos
interpretables pero no demasiado complejos, sin perder el objetivo principal de
este informe.

\subsection{Análisis descriptivo}

En esta sección se presentaran diversos gráficos para ayudar a entender un poco
mejor la población en estudio.

En la figura \ref{TAS_vs_tpo} se puede observar que en general la TAS se
mantiene constante (o con una muy leve pendiente decreciente) a través del
tiempo. Esto a simple vista no resultaría muy alentador, dado que el propósito
del tratamiento es disminuir la TAS a niveles más saludables. Sin embargo, como
se mencionó anteriormente, los pacientes no adhirieron al 100\% el tratamiento,
este efecto es el que estudiaremos más adelante. Cabe destacar que las
mediciones de los pacientes son equiespaciadas en el tiempo, las desviaciones
del eje en el tiempo se deben a una técnica llamada ``jitter'' que nos permite
mover levemente los puntos en el eje x para poder observar mejor la densidad de
los mismos. 

\begin{figure}[H]
	\centering
	\includegraphics[scale=0.5]{img/TAS_vs_tpo.png}
	\caption{TAS de cada paciente en cada tiempo}
	\label{TAS_vs_tpo}
\end{figure}

En la figura \ref{spaghetti} se pueden observar las trayectorias individuales
de 15 pacientes seleccionados al azar, las pendientes son muy similares entre
sí, sin embargo hay variación en la ordenada al origen.

\begin{figure}[H]
	\centering
	\includegraphics[scale=0.5]{img/spaghetti_plot.png}
	\caption{TAS a través del tiempo de 15 pacientes al azar}
	\label{spaghetti}
\end{figure}

Otro gráfico que resulta de interés es observar la evolución de la TAS a través
del tiempo pero sobre cada grupo de las covariables fijas, el resultado se
expresa en la figura \ref{TAS_with_covs}. Para las variables continuas, se
utilizó como punto de corte para segmentar en grupos la mediana de sus valores.
Podemos observar que, ya sea en mayor o en menor medida, estas covariables
parecen tener un efecto en la ordenada al origen pero no en la pendiente.

\begin{figure}[H]
	\centering
	\includegraphics[scale=0.4]{img/TAS_vs_tpo_with_covs.png}
	\caption{TAS a través del tiempo sobre grupos de covariables}
	\label{TAS_with_covs}
\end{figure}




\newpage
\nocite{*}
\renewcommand{\refname}{Bibliografía}
\bibliography{Bibliografia}

\end{document}