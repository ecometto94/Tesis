\documentclass[12pt]{article}
\usepackage[a4paper, left=1in, right=1in, top=1in, bottom=1in]{geometry}
                                % for page size and margin settings
\usepackage{graphicx}		    % for insert images
\usepackage[spanish]{babel} 	% for spanish titles
\usepackage[a4paper, left=1in, right=1in, top=1in, bottom=1in]{geometry} 
								% for page size and margin settings
\usepackage{mathtools}          % for greek math symbol formatting
\usepackage{enumitem}           % for control of 'enumerate' numbering
\usepackage{listings}           % for control of 'itemize' spacing
\usepackage{indentfirst}		% package to make first paragraph always indented
\usepackage{hyperref}           % page numbers and '\ref's become clickable
\usepackage{bm}					% for bold maths
\usepackage{hyperref}           % page numbers and '\ref's become clickable
\usepackage{setspace}			% for setting interline spacing
\usepackage{amsmath}			% for matrices
\renewcommand{\baselinestretch}{1.5}
\bibliographystyle{ieeetr}

% TITLE VARIABLES 

\def\thesistitle{Modelos longitudinales con covariables que varían en el tiempo}
\def\thesisauthorfirst{\textbf{Esteban Cometto}}
\def\thesissupervisorfirst{Noelia Castellana}
\def\thesissupervisorsecond{Cecilia Rapelli}
\def\thesisdate{\today}

%% OTHER USEFUL VARIABLES 

\def\npatients{560}
\def\fullcovname{adherencia al tratamiento farmacológico}
\def\covname{\emph{adherencia}}
\def\cvt{covariable que varía en el tiempo}
\def\xseqj{$X_{i1}, ..., X_{ij}$}
\def\xseqn{$X_{i1}, ..., X_{in_i}$}
\def\xseqjminus{$X_{i1}, ..., X_{ij-1}$}
\def\yseqj{$Y_{i1}, ..., Y_{ij}$}
\def\yseqn{$Y_{i1}, ..., Y_{in_i}$}
\def\yseqjminus{$Y_{i1}, ..., Y_{ij-1}$}

%% FOR PDF METADATA
\title{\thesistitle}
\author{\thesisauthorfirst\space\thesisauthorsecond}
\date{\thesisdate}

\begin{document}

\begin{titlepage}
    \newcommand{\HRule}{\rule{\linewidth}{0.5mm}}
	\center
	\textsc{\Large Universidad Nacional de Rosario}\\[.7cm]
	\includegraphics[width=25mm]{img/fceye-unr.png}\\[.5cm]
	\textsc{Facultad de Ciencias Económicas y Estadística}\\[0.5cm]
	\textsc{Anteproyecto de Tesina}
	
	\HRule \\[0.4cm]
	{ \huge \bfseries \thesistitle}\\[0.1cm]
	\HRule \\[.5cm]
	
	\begin{minipage}{0.6\textwidth}
	\large
	\emph{Autor:}	\thesisauthorfirst
	\end{minipage}
	\\[.6cm]
	\begin{minipage}{0.6\textwidth}
	\emph{Directora:} 	\thesissupervisorfirst \\[.2cm]
	\emph{Codirectora:} 	\thesissupervisorsecond
	\end{minipage}
	\\[4cm]
	\vfill
	{\large \thesisdate}\\
	\clearpage
\end{titlepage}

\newpage
\tableofcontents

\newpage
\section{Introducción}

%% Tesis mara catalano, silvia camats y redacción mia

Los datos longitudinales están conformados por mediciones repetidas de una misma variable realizadas a la misma unidad.
Estas mediciones surgen de observar unidades en diferentes ocasiones, es decir en diferentes momentos o condiciones
experimentales.

Dado que las mediciones repetidas son obtenidas de la misma unidad, los datos longitudinales están agrupados. Las
observaciones dentro de un mismo agrupamiento generalmente están correlacionadas positivamente. Por lo tanto, los
supuestos usuales acerca de la independencia entre las respuestas de cada unidad y la homogeneidad de variancias
frecuentemente no son válidos

El objetivo principal de estos estudios es estudiar los cambios en el tiempo y los factores que influencian el cambio.

Las ocasiones en las que se registran las mediciones repetidas no necesariamente serán iguales para todos los
individuos, por lo tanto se pueden obtener tanto estudios balanceados (todos los individuos tienen el mismo número de
mediciones durante un conjunto de ocasiones comunes) como desbalanceados (la secuencia de tiempos de observaciones no es
igual para todos los individuos). Otra característica de estos datos es que en ocasiones se pueden obtener valores
perdidos, obteniendo datos incompletos aunque se cuente con un estudio balanceado.

Los modelos mixtos permiten ajustar datos con estas particularidades, donde la
respuesta es modelada por una parte sistemática que está formada por una combinación de
características poblacionales que son compartidas por todas las unidades (efectos fijos), y una
parte aleatoria que está constituida por efectos específicos de cada unidad (efectos aleatorios)
y por el error aleatorio. Estos modelos permiten, además, hacer predicciones del perfil de una
unidad específica. La selección de la estructura de covariancia apropiada produce estimadores más eficientes
y consecuentemente, pruebas estadísticas más robustas para los efectos de interés

%% Chapter lalonde

Las covariables en los estudios longitudinales se pueden clasificar en dos categorias: fijas y variables en el tiempo.
Las diferencias entre estos tipos de covariables pueden llevar a diferentes intereses de investigación, diferentes tipos
de análisis y diferentes conclusiones.

Las covariables fijas son variables independientes que no tienen variación intra-sujeto, lo que significa que el valor
de la covariable no cambia para un individuo determinado en el estudio longitudinal. Este tipo de covariable se puede usar para
realizar comparaciones entre poblaciones y describir diferentes tendencias en el tiempo, pero no permite una relación
dinámica entre la covariable y la respuesta.

Las covariables variables en el tiempo (CVT) son variables independientes que contienen ambas variaciones, intra y entre
sujeto, lo que significa que el valor de la covariable cambia para un individuo determinado a lo largo del tiempo y
además puede cambiar para diferentes sujetos. Una CVT se puede usar para hacer comparaciones entre poblaciones, describir
tendencias en el tiempo y también la relación dinámica entre la CVT y la respuesta

Se puede ver que las CVT permiten diferentes tipos de relaciones y conclusiones que las covariables fijas. Por ejemplo,
una CVT puede estar involucrada en efectos acumulados para diferentes valores a través del tiempo (Fitzmaurice y Lard 1995).
Además, ciertas CVT transmiten diferente información que otras. Por ejemplo, covariables como la edad pueden cambiar a
través del tiempo, pero cambian de manera predecible. Por otro lado, covariables como la precipitación diaria pueden cambiar
a través del tiempo pero no pueden ser predecidas. En esos casos es importante considerar las relaciones entre la CVT y
la respuesta a través del tiempo.

%% Redacción mia

En el presente informe se cuenta con un programa de atención y control de pacientes hipertensos iniciado en el año 2014 en 
Rosario que realiza un seguimiento exhaustivo de \npatients{} pacientes. Este programa contempla: efectores no médicos
supervisados, tratamiento farmacológico genérico para la hipertensión y utilización de un algoritmo terapéutico sistematizado.
En cada visita se registran tantocaracterísticas de los pacientes, del tratamiento y de los valores de la tensión arterial.
En particular, se desea evaluar si la adherencia al tratamiento farmacológico influye en los valores de la tensión arterial
sistólica a lo largo del seguimiento. Como la variable “\fullcovname{}” es una CVT estocástica se evaluaran diferentes
enfoques para incluirla en un modelo longitudinal que pueda explicar el cambio en la tensión arterial sistólica media a
lo largo del tiempo.

Un aspecto a tener en cuenta en este trabajo es que, si bien contamos con mucha otra información para
obtener modelos que describan de mejor manera el comportamiento de la TAS, nos centraremos en modelos más simples con
respecto a las covariables fijas con el fin de no perder de vista la relación entre la variable respuesta y la CVT.

\newpage
\section{Objetivos}

\subsection{Objetivo Principal}

Profundizar en el estudio de propuestas metodológicas para utilizar la información obtenida de la \cvt{} dentro de un
modelo mixto.

\subsection{Objetivos Específicos}

\begin{itemize}
	\item Específicar distintos tipos de CVT
	\item Transformaciones a realizar sobre la CVT antes de incluirla al modelo, incluyendo conversión a covariable fija
	\item Consideraciones sobre interpretación de los parámetros sobre las CVT
	\item Indagar sobre feedback entre la CVT y la variable respuesta
\end{itemize}

\newpage
\section{Metodología}

\subsection{Los Datos Longitudinales}

Los datos longitudinales están conformados por mediciones repetidas de una misma variable realizadas a la misma unidad.
Estas mediciones surgen de observar unidades en diferentes ocasiones, es decir en diferentes momentos o condiciones
experimentales.

Dado que las mediciones repetidas son obtenidas de la misma unidad, los datos longitudinales están agrupados.
Las observaciones dentro de un mismo agrupamiento generalmente están correlacionadas positivamente.

El objetivo principal de estos estudios es estudiar los cambios en el tiempo y los factores que influencian el cambio. 

Las ocasiones en las que se registran las mediciones repetidas no necesariamente serán iguales para todos los individuos,
por lo tanto se pueden obtener tanto estudios balanceados (todos los individuos tienen el mismo número de mediciones
durante un conjunto de ocasiones comunes) como desbalanceados (la secuencia de tiempos de observaciones no es igual
para todos los individuos). Otra característica de estos datos es que en ocasiones se pueden obtener valores perdidos,
obteniendo datos incompletos aunque se cuente con un estudio balanceado.

Con el fin de simplificar la notación, se asumirá que los tiempos de medición son los mismos para todas las unidades y
que no hay datos faltantes.

Se obtiene una muestra de $N$ unidades cada una con $n$ mediciones repetidas de la variable en estudio, observadas en los
tiempos $t_1, t_2, ..., t_n$, siendo entonces el número total de observaciones $N^*=Nn$. Se le llama $Y_{ij}$ a la
medición sobre la unidad $i$ en la ocasión $j$, con $i=1, ..., N; j=1, ..., n$

Asociadas a cada unidad se observan las covariables $X_{ij}$, de las cuales existen dos tipos: variables en el tiempo
(estocásticas) e invariables en el tiempo (estacionarias)

Existen estudios empíricos que llevan a pensar que existen tres fuentes potenciales de variabilidad que influyen
sobre la correlación entre medidas repetidas:

\begin{itemize}
	\item \emph{Heterogeneidad entre las unidades:} Refleja la propensión natural de las unidades a responder.
	Los individuos tienen diferentes reacciones frente a los mismos estímulos.
	\item \emph{Variación biológica intra-unidad:} Se piensa que la secuencia de medidas repetidas de una unidad tiene
	un comportamiento determinado, que produce que las mediciones más cercanas sean más parecidas.
	\item \emph{Error de medición:} Surge debido a los errores de medida, se puede disminuir usando instrumentos
	de medición más precisos
\end{itemize}

Estas tres fuentes de variación pueden clasificarse en \emph{"variabilidad entre"}, es decir, la variación entre
las unidades (heterogeneidad entre unidades) y \emph{"variabilidad intra"}, es decir, la variación entre las mediciones
de las misma unidad (variación biológica intra-unidad y error de medición)

Dado que, como se mencionó anteriormente, las mediciones están correlacionadas sí, si se utilizaran las técnicas habituales
basadas en la independencia entre mediciones, los errores estándares nominales van a ser incorrectos, lo cual nos llevaría
a inferencias incorrectas sobre los parámetros del modelo. En base a esto, surgen técnicas que consideran esa correlación
modelando los datos considerando la modelación de dos estructuras: la parte media y la estructura de covariancia.

\subsection{Análisis exploratorio}

Antes de ajustar algún modelo, lo primero siempre es realizar un análisis exploratorio para estudiar cómo se comportan
los datos. A continuación se presentan técnicas gráficas para cada estructura.

\emph{Evaluación de la parte media}

\begin{itemize}
	\item \emph{Perfil individual:} Consiste en un gráfico de dispersión en el cual se representan las respuestas vs las
	ocasiones. Cada respuesta tiene un punto y se une con un segmento los puntos de la misma unidad. Sirven para detectar
	si hay mucha variabilidad entre y dentro de las unidades y si hay valores atípicos.
	\item \emph{Perfiles promedio por grupo:} En general son más informativos. Para cada tiempo calculamos un promedio
	para cada grupo y luego se unen los puntos. Permiten ver la tendencia de las variables a través de las ocasiones.
	Se superponen en un mismo gráfico los perfiles promedio de cada grupo
\end{itemize}

\emph{Evaluación de la estructura de covariancia}

\begin{itemize}
	\item \emph{Matriz de diagrama de dispersión:} Para cada par de ocasiones se grafican los valores esperados de la
	respuesta y todos estos gráficos se acomodan dentro de una matriz. En general se utiliza cuando las ocasiones son las
	mismas para todas las unidades.
	\item \emph{Gráfico de Draftman:} Es similar al gráfico anterior pero utilizando variables estandarizadas.
	La utilización de la variable respuesta estandarizada ayuda a eliminar la variabilidad de los datos asociada con
	diferencias en las medias y variancias en el tiempo, permitiendo visualizar más claramente el patrón de correlación.
	\item \emph{Gráfico PRISM (Partial Regression on Intervenors Scatterplot Matrix):} Utilizando la variable estandarizada
	se crea una matriz de gráficos de dispersión. En la primera diagonal se encuentran gráficos de dispersión entre la
	variable respuesta en los tiempos $t_j$ y $t_{j+1}$. Luego, en la k-ésima diagonal, se obtienen gráficos de regresión
	parcial de las respuestas en los tiempos $t_j$ y $t_{j+k}$, ajustadas por las respuestas en los tiempos intermedios.
	Estos gráficos permiten ver con mayor claridad ciertos tipos de estructuras seriales que se dan entre las
	medidas repetidas.
	\item \emph{Correlograma:} Representa las características que existen entre las respuestas de los individuos de
	cada grupo en tiempos que están separados una cantidad de periodos. Permite analizar cómo evoluciona la correlación
	a medida que aumenta el número de rezagos.
	\item \emph{Semivariograma:} Cuando los datos están desbalanceados, el semivariograma permite distinguir las 3 fuentes
	de variabilidad. Después de haber estimado un modelo, el mismo permite confirmar si la estructura de correlación es
	adecuada. El semivariograma se define como una función:
	\[ \gamma(u) = \frac{1}{2} E[(\varepsilon _{ij} - \varepsilon_{ij'})^2] \]
	\[ u_{ijj'} = |t_{ij} - t_{ij'}| \]
	\[ \widehat{\gamma(u)} = v_{ijj'} = \frac{1}{2} (r_{ij} - r_{ij'}) \]
	donde $r_{ij}$ y $r_{ij'}$ son los residuos estandarizados obtenidos después de ajustar un modelo de regresión
	considerando las observaciones independientes.

	Se va a obtener un gráfico donde la variabilidad total va a estar dividida en 3 partes. Si la curva no empieza en
	cero significa que hay error de medición, si tiene pendiente quiere decir que hay un error debido a una causa
	biológica (correlación serial) y si la misma no llega a la variancia total significa que se debe explicar la
	variabilidad entre.
\end{itemize}

\subsection{Modelo lineal general para datos longitudinales}

Si se piensa que existe una tendencia en el tiempo de las respuestas, y esta tendencia se puede expresar como una función,
se puede escribir o representar a las medidas repetidas de una unidad en un vector $Y_i$. Entonces, un modelo lineal para
representar la evolución en el tiempo va a ser:
\[ Y_i = X_i\beta + \varepsilon_i; i = 1, ..., N; Y_i = (Y_{i1}, Y_{i2}, ..., Y_{in_{i}})' \]
Donde:

$Y_{ij}$: respuesta obtenida de la i-ésima unidad en la ocasión $t_{ij}$.

$X_i$: matriz de diseño de la i-ésima unidad, de dimensión $(n_i*p)$

$\beta $: vector de parámetros de dimensión $(p*1)$

$\varepsilon_i$: vector de errores aleatorios de la i-ésima unidad, de dimensión $(n_i*1)$, este mismo representa todas
las fuentes de variabilidad de los datos longitudinales
\[ \varepsilon_i ~ N_{n_i}(0, \varSigma_i(\theta )) \]

$\theta$: vector de parámetros desconocidos de covariancia, de dimensión $(q*1)$

\subsection{Modelación de la estructura de covariancia}

Al tenerse tantos parámetros de variancia $(n)$ y covariancia $n(n-1)/2$ para estimar, se proponen modelos específicos
para la estructura de correlación. Se trata de elegir una estructura que no tenga tantos parámetros. Sin embargo, se debe
tener cuidado de no seleccionar estructuras demasiado parcas con las que se pierda información.

La matriz de covariancia de cada unidad va a ser función de $\theta$. El número de parámetros de este vector depende de la
estructura de la matriz.

A continuación se mencionan algunas estructuras que se pueden utilizar, se llamará $R$ a la matriz de correlaciones

\begin{itemize}
	\item \emph{Arbitraria o no estructurada (datos balanceados):} Considera variancias y covariancias distintas entre
	las mediciones repetidas. Siendo $\sigma_{jj'} = Cov(Y_{ij}Y_{ij'})$ se expresa como:
	\[
	\varSigma = 
	\begin{bmatrix}
		\sigma_{1}^2 & \sigma_{12}  & \dots  & \sigma_{1n} \\
		\sigma_{21}  & \sigma_{2}^2 & \dots  & \sigma_{2n} \\
		\vdots 		 & \vdots	    & \ddots & \vdots	   \\
		\sigma_{n1}  & \sigma_{n2}  & \dots  & \sigma_{n}^2
	\end{bmatrix}
	\]
	La ausencia de restricciones hace que haya que estimar una gran cantidad de parámetros
	\item \emph{Simetria compuesta (datos balanceados o no balanceados):} La correlación entre pares de observaciones
	es la misma, sin importar la cantidad de rezagos entre ellas, $Corr(Y_{ij}, Y_{ik}) = \rho $ para todo $j \neq k$
	\[
	R_i =
	\begin{bmatrix}
		1      & \rho   & \dots  & \rho \\
		\rho   & 1      & \dots  & \rho \\
		\vdots & \vdots	& \ddots & \vdots \\
		\rho   & \rho   & \dots  & 1
	\end{bmatrix}
	\]
	\item \emph{Toeplitz (datos equiespaciados):} Se plantea para que cualquier par de respuestas que estén igualmente
	separadas en el tiempo la correlación es la misma, $Corr(Y_{ij},Y_{ij+k}) = \rho_{k}$ para todo $j$ y $k$.
	\[
	R_i =
	\begin{bmatrix}
		1      & \rho_1 	& \dots  & \rho_n \\
		\rho_1 & 1     		& \dots  & \rho_{n-1} \\
		\vdots & \vdots		& \ddots & \vdots \\
		\rho_n & \rho_{n-1} & \dots  & 1
	\end{bmatrix}
	\]
	\item \emph{Autorregresiva de primer orden (datos equiespaciados):} Es un caso especial de la estructura anterior,
	en la que $Corr(Y_{ij}, Y_{ij+k}) = \rho^k$. Esta estructura asume que la correlación entre medidas repetidas
	disminuye a medida que aumenta el número de rezagos entre ellas.
	\[
	R_i =
	\begin{bmatrix}
		1      & \rho 	    & \dots  & \rho^n \\
		\rho   & 1     		& \dots  & \rho^{n-1} \\
		\vdots & \vdots		& \ddots & \vdots \\
		\rho^n & \rho^{n-1} & \dots  & 1
	\end{bmatrix}
	\]
	\item \emph{Markov (datos no equiespaciados):} Es una generalización de la estructura autorregresiva para mediciones
	no equiespaciadas. $Corr(Y_{ij}, Y_{ij'}) = \rho^{d_{jj'}}$, donde $d_{jj'} = |t_{ij} - t_{ij'}|$ para todo $j \neq j'$.
	\[
	R_i =
	\begin{bmatrix}
		1      		   & \rho^{d_{12}} 	    & \dots  & \rho^{d_{1n}} \\
		\\rho^{d_{21}} & 1     		& \dots  & \rho^{n-1} \\
		\vdots 		   & \vdots		& \ddots & \vdots \\
		\rho^n 		   & \rho^{n-1} & \dots  & 1
	\end{bmatrix}
	\]
\end{itemize}


\end{document}