\documentclass[12pt]{article}
\usepackage[a4paper, left=1in, right=1in, top=1in, bottom=1in]{geometry}
                                % for page size and margin settings
\usepackage{graphicx}		    % for insert images
\usepackage[spanish]{babel} 	% for spanish titles
\usepackage[a4paper, left=1in, right=1in, top=1in, bottom=1in]{geometry} 
								% for page size and margin settings
\usepackage{mathtools}          % for greek math symbol formatting
\usepackage{enumitem}           % for control of 'enumerate' numbering
\usepackage{listings}           % for control of 'itemize' spacing
\usepackage{indentfirst}		% package to make first paragraph always indented
\usepackage{hyperref}           % page numbers and '\ref's become clickable
\usepackage{bm}					% for bold maths
\usepackage{hyperref}           % page numbers and '\ref's become clickable
\usepackage{setspace}			% for setting interline spacing
\renewcommand{\baselinestretch}{1.5}
\bibliographystyle{ieeetr}

% TITLE VARIABLES 

\def\thesistitle{Modelos longitudinales con covariables que varían en el tiempo}
\def\thesisauthorfirst{\textbf{Esteban Cometto}}
\def\thesissupervisorfirst{Noelia Castellana}
\def\thesissupervisorsecond{Cecilia Rapelli}
\def\thesisdate{\today}

%% OTHER USEFUL VARIABLES 

\def\npatients{560}
\def\fullcovname{adherencia al tratamiento farmacológico}
\def\covname{\emph{adherencia}}
\def\cvt{covariable que varía en el tiempo}
\def\xseqj{$X_{i1}, ..., X_{ij}$}
\def\xseqn{$X_{i1}, ..., X_{in_i}$}
\def\xseqjminus{$X_{i1}, ..., X_{ij-1}$}
\def\yseqj{$Y_{i1}, ..., Y_{ij}$}
\def\yseqn{$Y_{i1}, ..., Y_{in_i}$}
\def\yseqjminus{$Y_{i1}, ..., Y_{ij-1}$}

%% FOR PDF METADATA
\title{\thesistitle}
\author{\thesisauthorfirst\space\thesisauthorsecond}
\date{\thesisdate}

\begin{document}

\begin{titlepage}
    \newcommand{\HRule}{\rule{\linewidth}{0.5mm}}
	\center
	\textsc{\Large Universidad Nacional de Rosario}\\[.7cm]
	\includegraphics[width=25mm]{img/fceye-unr.png}\\[.5cm]
	\textsc{Facultad de Ciencias Económicas y Estadística}\\[0.5cm]
	\textsc{Anteproyecto de Tesina}
	
	\HRule \\[0.4cm]
	{ \huge \bfseries \thesistitle}\\[0.1cm]
	\HRule \\[.5cm]
	
	\begin{minipage}{0.6\textwidth}
	\large
	\emph{Autor:}	\thesisauthorfirst
	\end{minipage}
	\\[.6cm]
	\begin{minipage}{0.6\textwidth}
	\emph{Directora:} 	\thesissupervisorfirst \\[.2cm]
	\emph{Codirectora:} 	\thesissupervisorsecond
	\end{minipage}
	\\[4cm]
	\vfill
	{\large \thesisdate}\\
	\clearpage
\end{titlepage}

\newpage
\tableofcontents

\newpage
\section{Introducción}

%% Tesis mara catalano, silvia camats y redacción mia

Los datos longitudinales están conformados por mediciones repetidas de una misma variable realizadas a la misma unidad.
Estas mediciones surgen de observar unidades en diferentes ocasiones, es decir en diferentes momentos o condiciones
experimentales.

Dado que las mediciones repetidas son obtenidas de la misma unidad, los datos longitudinales están agrupados. Las
observaciones dentro de un mismo agrupamiento generalmente están correlacionadas positivamente. Por lo tanto, los
supuestos usuales acerca de la independencia entre las respuestas de cada unidad y la homogeneidad de variancias
frecuentemente no son válidos

El objetivo principal de estos estudios es estudiar los cambios en el tiempo y los factores que influencian el cambio.

Las ocasiones en las que se registran las mediciones repetidas no necesariamente serán iguales para todos los
individuos, por lo tanto se pueden obtener tanto estudios balanceados (todos los individuos tienen el mismo número de
mediciones durante un conjunto de ocasiones comunes) como desbalanceados (la secuencia de tiempos de observaciones no es
igual para todos los individuos). Otra característica de estos datos es que en ocasiones se pueden obtener valores
perdidos, obteniendo datos incompletos aunque se cuente con un estudio balanceado.

Los modelos mixtos permiten ajustar datos con estas particularidades, donde la
respuesta es modelada por una parte sistemática que está formada por una combinación de
características poblacionales que son compartidas por todas las unidades (efectos fijos), y una
parte aleatoria que está constituida por efectos específicos de cada unidad (efectos aleatorios)
y por el error aleatorio. Estos modelos permiten, además, hacer predicciones del perfil de una
unidad específica. La selección de la estructura de covariancia apropiada produce estimadores más eficientes
y consecuentemente, pruebas estadísticas más robustas para los efectos de interés

%% Chapter lalonde

Las covariables en los estudios longitudinales se pueden clasificar en dos categorias: fijas y variables en el tiempo.
Las diferencias entre estos tipos de covariables pueden llevar a diferentes intereses de investigación, diferentes tipos
de análisis y diferentes conclusiones.

Las covariables fijas son variables independientes que no tienen variación intra-sujeto, lo que significa que el valor
de la covariable no cambia para un individuo determinado en el estudio longitudinal. Este tipo de covariable se puede usar para
realizar comparaciones entre poblaciones y describir diferentes tendencias en el tiempo, pero no permite una relación
dinámica entre la covariable y la respuesta.

Las covariables variables en el tiempo (CVT) son variables independientes que contienen ambas variaciones, intra y entre
sujeto, lo que significa que el valor de la covariable cambia para un individuo determinado a lo largo del tiempo y
además puede cambiar para diferentes sujetos. Una CVT se puede usar para hacer comparaciones entre poblaciones, describir
tendencias en el tiempo y también la relación dinámica entre la CVT y la respuesta

Se puede ver que las CVT permiten diferentes tipos de relaciones y conclusiones que las covariables fijas. Por ejemplo,
una CVT puede estar involucrada en efectos acumulados para diferentes valores a través del tiempo (Fitzmaurice y Lard 1995).
Además, ciertas CVT transmiten diferente información que otras. Por ejemplo, covariables como la edad pueden cambiar a
través del tiempo, pero cambian de manera predecible. Por otro lado, covariables como la precipitación diaria pueden cambiar
a través del tiempo pero no pueden ser predecidas. En esos casos es importante considerar las relaciones entre la CVT y
la respuesta a través del tiempo.

%% Redacción mia

En el presente informe se cuenta con un programa de atención y control de pacientes hipertensos iniciado en el año 2014 en 
Rosario que realiza un seguimiento exhaustivo de \npatients{} pacientes. Este programa contempla: efectores no médicos
supervisados, tratamiento farmacológico genérico para la hipertensión y utilización de un algoritmo terapéutico sistematizado.
En cada visita se registran tantocaracterísticas de los pacientes, del tratamiento y de los valores de la tensión arterial.
En particular, se desea evaluar si la adherencia al tratamiento farmacológico influye en los valores de la tensión arterial
sistólica a lo largo del seguimiento. Como la variable “\fullcovname{}” es una CVT estocástica se evaluaran diferentes
enfoques para incluirla en un modelo longitudinal que pueda explicar el cambio en la tensión arterial sistólica media a
lo largo del tiempo.

Un aspecto a tener en cuenta en este trabajo es que, si bien contamos con mucha otra información para
obtener modelos que describan de mejor manera el comportamiento de la TAS, nos centraremos en modelos más simples con
respecto a las covariables fijas con el fin de no perder de vista la relación entre la variable respuesta y la CVT.

\end{document}