
\documentclass[spanish]{article}
\usepackage[a4paper]{geometry}  % for page size and margin settings
\geometry{left=3.5cm, right=1cm, top=3cm, bottom=1.5cm}
\usepackage{graphicx}		    % for insert images
\usepackage[es-tabla]{babel} 	% for spanish titles
\usepackage{mathtools}          % for greek math symbol formatting
\usepackage{enumitem}           % for control of 'enumerate' numbering
\usepackage{listings}           % for control of 'itemize' spacing
\usepackage{indentfirst}		% package to make first paragraph always indented
\usepackage{hyperref}           % page numbers and '\ref's become clickable
\usepackage{bm}					% for bold maths
\usepackage{setspace}			% for setting interline spacing
\usepackage{amsmath}			% for matrices
\usepackage{tikz} 				% for graphs
\usepackage{multirow}			% for tables
\usepackage{float}				% to manually select placement of tables
\usepackage{autobreak}          % to automatically fit equations to page width


\usetikzlibrary{babel}		    % for draw arrows in tikz using babel spanish
\doublespacing					% for use double interline spacing
\bibliographystyle{ieeetr}
\numberwithin{figure}{subsection}
\numberwithin{equation}{subsection}
\numberwithin{table}{subsection}

% TITLE VARIABLES 

\def\thesistitle{Incorporación de covariables que varían en el tiempo a un modelo mixto}
\def\thesisauthorfirst{\textbf{Esteban Cometto}}
\def\thesissupervisorfirst{Noelia Castellana}
\def\thesissupervisorsecond{Cecilia Rapelli}
\def\thesisdate{\today}

%% OTHER USEFUL VARIABLES 

\def\npatients{560}
\def\fullcovname{adherencia al tratamiento farmacológico}
\def\covname{\textit{adherencia}}
\def\cvt{covariable que varía en el tiempo}
\def\xseqj{$X_{i1}, ..., X_{ij}$}
\def\xseqn{$X_{i1}, ..., X_{in}$}
\def\xseqjminus{$X_{i1}, ..., X_{ij-1}$}
\def\yseqj{$Y_{i1}, ..., Y_{ij}$}
\def\yseqn{$Y_{i1}, ..., Y_{in}$}
\def\yseqjminus{$Y_{i1}, ..., Y_{ij-1}$}

%% FOR PDF METADATA
\title{\thesistitle}
\author{\thesisauthorfirst\space\thesisauthorsecond}
\date{\thesisdate}

\begin{document}

\begin{titlepage}
    \newcommand{\HRule}{\rule{\linewidth}{0.5mm}}
	\center
	\textsc{\Large Universidad Nacional de Rosario}\\[.7cm]
	\includegraphics[width=25mm]{img/fceye-unr.png}\\[.5cm]
	\textsc{Facultad de Ciencias Económicas y Estadística}\\[0.5cm]
	\textsc{Anteproyecto de Tesina}
	
	\HRule \\[0.4cm]
	{ \huge \bfseries \thesistitle}\\[0.1cm]
	\HRule \\[.5cm]
	
	\begin{minipage}{0.6\textwidth}
	\large
	\textit{Autor:}	\thesisauthorfirst
	\end{minipage}
	\\[.6cm]
	\begin{minipage}{0.6\textwidth}
	\textit{Directora:} 	\thesissupervisorfirst \\[.2cm]
	\textit{Codirectora:} 	\thesissupervisorsecond
	\end{minipage}
	\\[4cm]
	\vfill
	{\large \thesisdate}\\
	\clearpage
\end{titlepage}

\newpage
\tableofcontents

\newpage
\section{Introducción}

Los datos longitudinales están conformados por mediciones repetidas sobre una
unidad, las cuales pueden surgir por ser medidas en diferentes momentos o
condiciones. Su principal objetivo es estudiar los cambios en el tiempo y los
factores que influencian el cambio. 

Los modelos mixtos permiten ajustar datos con estas características, donde la
respuesta se modela por una parte sistemática que está compuesta por una
combinación de características poblacionales que son compartidas por todas las
unidades (efectos fijos), y una parte aleatoria que está constituida por efectos
específicos de cada unidad (efectos aleatorios) y por el error aleatorio, las
cuales reflejan las múltiples fuentes de heterogeneidad y correlación entre y
dentro de las unidades.

En estos modelos pueden incorporarse covariables. Las mismas se pueden
clasificar en 2 categorías: covariables no variables en el tiempo (CNVT) y
covariables variables en el tiempo (CVT). La naturaleza diferente de estas
covariables conduce a considerar distintos enfoques para cada una de ellas en el
análisis.

Las CNVT son variables independientes que no tienen variación intra-unidad, es
decir que el valor de la covariable no cambia para una unidad determinada en
el estudio longitudinal. Este tipo de covariables se pueden utilizar para
realizar comparaciones entre poblaciones y describir diferentes tendencias en el
tiempo.

Las CVT son variables independientes que contienen ambas variaciones, intra y
entre unidad, es decir que el valor de la covariable cambia para una unidad
determinada a lo largo del tiempo y además puede cambiar para diferentes
unidades. Este tipo de covariables tienen los mismos usos que las CNVT, y además
permiten describir la relación dinámica entre la CVT y la respuesta. Sin
embargo, esta relación puede estar confundida por valores anteriores y/o
posteriores de la covariable y en consecuencia esto puede conducir a inferencias
engañosas sobre los parámetros del modelo. Esta tesina realiza una introducción
a la problemática de incorporar covariables que varían en el tiempo en modelos
mixtos para datos longitudinales, presentando diferentes definiciones de las
mismas y enfoques metodológicos.

Estos conceptos se aplican a un conjunto de datos que surge del programa de
atención y control de pacientes hipertensos de Fundación ECLA llevado a cabo en
Rosario durante el período 2014-2019. Este estudio observacional realizó un
seguimiento de un grupo de pacientes hipertensos registrando en cada visita el
tratamiento farmacológico dado al paciente, los valores de la tensión arterial
sistólica (TAS) y la adherencia a dicho tratamiento entre otras características.
Uno de los objetivos que persiguió este estudio fue evaluar si la adherencia al
tratamiento influye en los valores de la TAS a lo largo del seguimiento. Como la
variable adherencia es una CVT, se presentarán diferentes enfoques para
incluirla en un modelo longitudinal mixto que pueda explicar el cambio en la
tensión arterial sistólica media a lo largo del tiempo.

\newpage
\section{Objetivos}

\subsection{Objetivo Principal}

Presentar diferentes propuestas metodológicas para la incorporación de
covariables que varían con el tiempo en modelos mixtos para datos
longitudinales.

\subsection{Objetivos Específicos}

\begin{itemize}
	\item Definir los tipos de covariables existentes.
	\item Describir propuestas de incorporación de covariables que varían en el
	tiempo en los modelos mixtos.
	\item Aplicar los conceptos vistos al programa de
	atención y control de pacientes hipertensos de Fundación ECLA.
\end{itemize}

\newpage
\section{Datos Longitudinales}

Los datos longitudinales están conformados por mediciones repetidas de una misma
variable realizadas a la misma unidad en diferentes momentos o condiciones
experimentales.

Dado que las mediciones repetidas son obtenidas de la misma unidad, los datos
longitudinales están agrupados. Las observaciones dentro de un mismo
agrupamiento generalmente están correlacionadas positivamente. Por lo tanto, los
supuestos usuales de independencia y homogeneidad de variancias no son válidos

Existen tres fuentes potenciales de variabilidad que influyen sobre la
correlación entre medidas repetidas:

\begin{itemize}
	\item \textit{Heterogeneidad entre las unidades:} Refleja la propensión
	natural de las unidades a responder. Las unidades tienen diferentes
	reacciones frente a los mismos estímulos.
	\item \textit{Variación biológica intra-unidad:} Se espera que la secuencia
	de medidas repetidas de una unidad tenga un comportamiento determinado, que
	produce que las mediciones más cercanas sean más parecidas entre sí que las
	más alejadas.
	\item \textit{Error de medición:} Errores aleatorios asociados al proceso de
	medición.
\end{itemize}

Estas tres fuentes de variación pueden clasificarse en \textit{``variabilidad
entre unidades''} (heterogeneidad entre unidades) y \textit{``variabilidad
intra unidades''} (variación biológica intra-unidad y error de medición)

Dado que estas fuentes de variabilidad introducen correlación entre las
mediciones repetidas, no se pueden utilizar las técnicas habituales, ya que
llevarían a inferencias incorrectas sobre los parámetros del modelo.

Con el fin de simplificar la notación, se asumirá que los tiempos de medición
son los mismos para todas las unidades y que no hay datos faltantes.

Sean $Y_{ij}$ el valor de la variable respuesta y $X_{ij}$ las covariables,
medidas sobre la unidad $i$ en la ocasión $j$, se obtiene una muestra de $N$
unidades cada una con $n$ mediciones repetidas de la variable en estudio,
observadas en los tiempos $t_1, t_2, ..., t_n$, siendo entonces el número total
de observaciones $N^*=Nn$, con $i=1, ..., N; j=1, ..., n$.

Se asume que $Y_{ij}$ y $X_{ij}$ son simultáneamente medidas. Esto quiere decir
que en un análisis de corte transversal, $Y_{ij}$ y $X_{ij}$ se correlacionan
directamente. Sin embargo, para un análisis longitudinal se debe asumir que
existe un orden pre-establecido:
$(X_{i1}, Y_{i1}), (X_{i2}, Y_{i2}), ..., (X_{in}, Y_{in})$

\section{Covariables en datos longitudinales}

En los estudios longitudinales, las variables independientes pueden ser
clasificadas en dos categorías: CNVT y CVT. La diferencia entre
ellas puede conducir a diferentes enfoques de análisis así como también a
diferentes conclusiones.

Tanto las CNVT como las CVT se pueden utilizar para realizar comparaciones entre
poblaciones y describir diferentes tendencias a lo largo del tiempo. Sin
embargo, sólo las CVT permiten describir una relación dinámica entre la
covariable y la variable respuesta.

\subsection{Covariables fijas en el tiempo}

Las CNVT son variables independientes que no presentan variación intra-unidad,
es decir, los valores de estas covariables no cambian a lo largo del estudio
para una unidad en particular.

Éstas covariables pueden ser fijas por naturaleza (por ejemplo, el sexo
biológico de una persona o el grupo de tratamiento) o pueden ser covariables
basales (es decir, medidas al inicio del estudio). Las covariables basales son
fijas por definición pero pueden ser variables en el tiempo por naturaleza, por
ejemplo, la edad varia en el tiempo pero la edad basal es fija.

\subsection{Covariables variables en el tiempo}

Las CVT son variables independientes que incluyen tanto la variación
intra-unidad como la variación entre-unidad. Esto significa que, para una unidad
en particular, el valor de la covariable cambia a través del tiempo y puede
cambiar también entre diferentes unidades. Por ejemplo, el valor del colesterol
o la condición de fumador (si/no).

A continuación se describen diferentes tipos de CVT.

\subsubsection{Covariables estocásticas y no estocásticas}

Las CVT pueden clasificarse en estocásticas y no estocásticas. Las CVT no
estocásticas son covariables que varían sistemáticamente a través del tiempo
pero son fijas por diseño del estudio o bien su valor puede predecirse. En
cambio, las CVT estocásticas son covariables que varían aleatoriamente a través
del tiempo, es decir, los valores en cualquier ocasión no pueden ser estimados
ya que son gobernados por un mecanismo aleatorio. Ejemplos de las primeras son:
tiempo desde la visita basal o edad. Ejemplos de las segundas son: valor del
colesterol, ingesta de alcohol (si/no), ingesta de grasas, etc.

\subsubsection{Covariables exógenas y endógenas}
\label{seccion_de_exogeneidad}

Otra clasificación de las CVT es en exógenas y endógenas.

\paragraph{Covariables exógenas} \mbox{} \\

Una CVT se define como exógena, respecto a la variable respuesta, si el valor de
la covariable en un determinado momento es condicionalmente independiente de
todos los valores precedentes de la variable respuesta (Diggle et al., 2002).
Formalmente, para la unidad $i$ en la ocasión $j$:  

\begin{equation}
	\label{exogeneidad}
	f(X_{ij}|X_{i1}, ..., X_{ij-1}, Y_{i1}, ..., Y_{ij}) =
	f(X_{ij}|X_{i1}, ..., X_{ij-1})
\end{equation}

Y en consecuencia:

\begin{equation}
	\label{exogeneidad debil}
	E(Y_{ij}|X_{i1}, ..., X_{in}) = E(Y_{ij}|X_{i1}, ..., X_{ij})
\end{equation}

Esta definición implica que la respuesta en cualquier momento puede depender de
los valores previos de la variable respuesta y de la CVT, pero será
independiente de todos los demás valores de la covariable. Por ejemplo, en un
estudio longitudinal en donde se evalúa si el la cantidad de actividad física
(variable explicativa) está asociada al nivel de glucosa en sangre (variable
respuesta), es de esperar que la cantidad de actividad física en una determinada
ocasión, afecte a niveles posteriores de nivel de glucosa en sangre, pero no se
espera que dependa de los niveles glucosa en sangre observados previamente en el
sujeto afecten a la cantidad de actividad física.

Es posible examinar empíricamente la suposición de que una CVT es exógena al
considerar modelos de regresión para la dependencia de $X_{ij}$ en
$Y_{i1}, ..., Y_{ij-1}$ (o en alguna función conocida de
$Y_{i1}, ..., Y_{ij-1}$) y $X_{i1}, ..., X_{ij-1}$ (o en alguna función conocida
de $X_{i1}, ..., X_{ij-1}$). La ausencia de cualquier relación entre $X_{ij}$ y
$Y_{i1}, ..., Y_{ij-1}$, dado el perfil de la covariable anterior
$X_{i1}, ..., X_{ij-1}$, proporciona soporte para la validez de la suposición de
que la CVT es exógena.

Cuando se puede asumir que las CVT son exógenas con respecto a la variable
respuesta, se puede dar una interpretación causal a los parámetros de regresión.

\paragraph{Covariables endógenas} \mbox{} \\

Una CVT que no es exógena se define como endógena. Una variable endógena es una
variable estocásticamente relacionada con otros factores medidos en el estudio.
Esta también puede definirse como una variable generada por un proceso
estocástico relacionado con el individuo en estudio. En otras palabras, las CVT
endógenas están asociadas con un efecto individual y, a menudo, pueden
explicarse por otras variables en el estudio. Cuando el proceso estocástico de
una CVT endógena puede ser (al menos parcialmente) explicado por la variable
respuesta, se dice que hay \textit{feedback} entre la respuesta y la CVT
endógena. Por ejemplo, cuando se evalúa si la cantidad de actividad física está
asociada al nivel de glicemia. El nivel de actividad física en un determinado
momento puede estar (o no) asociado a niveles previos y también puede estar
asociado a valores previos de glicemia (un paciente con valor de glicemia alto
en una visita puede decidir aumentar su nivel de actividad fística para ver si
este valor se reduce).

\section{Modelo lineal mixto}

Los modelos lineales mixtos se utilizan habitualmente para analizar los datos
longitudinales, debido a que permiten modelar las distintas fuentes de
variabilidad presentes en los mismos.

En estos modelos, la respuesta media se modela como una combinación de
características poblacionales que son comunes a todos los individuos (efectos
fijos) y efectos específicos de la unidad que son únicos de ella (efectos
aleatorios).

El modelo lineal mixto para la unidad $i$ se puede expresar en forma matricial
como:

\[
	\bm{Y}_i = \bm{X}_i\bm{\beta} + \bm{Z}_i\bm{b}_i + \bm{\varepsilon}_i;
	\quad i = 1, ..., N;
\]

Donde:

\begin{itemize}
	\item $\bm{Y}_i$: Vector de la variable respuesta de la i-ésima unidad, de
	dimensión $(n \times 1)$, siendo $\bm{Y}_i = (Y_{i1}, Y_{i2}, ... Y_{in})'$
	\item $\bm{X}_i$: Matriz de diseño de la i-ésima unidad, que caracteriza la
	parte sistemática de la respuesta, de dimensión $(n \times p)$
	\item $\bm{\beta}$: Vector de parámetros de dimensión $(p \times 1)$
	\item $\bm{Z}_i$: Matriz de diseño de la i-ésima unidad, que caracteriza la
	parte aleatoria de la respuesta, de dimensión $(n \times k)$
	\item $\bm{b}_i$: Vector de efectos aleatorios de la i-ésima unidad, de
	dimensión $(k \times 1)$
	\item $\bm{\varepsilon}_i$: Vector de errores aleatorios de la i-ésima unidad,
	de dimensión $(n \times 1)$
\end{itemize}

Se supone que $\bm{\varepsilon}_i$ y $\bm{b}_i$ son independientes.

\[ \bm{\varepsilon}_i \sim N_{n}(0, \bm{R}_i) \quad \bm{b}_i \sim N_k(0, \bm{D}) \]

$\bm{D}$ y $\bm{R}_i$ son las matrices de variancias y covariancias de los
vectores $\bm{b}_i$ y $\bm{\varepsilon}_i$ respectivamente. A partir de este
modelo se obtiene:

\begin{itemize}
	\item $E(\bm{Y}_i/\bm{b}_i) = \bm{X}_i\bm{\beta} + \bm{Z}_i\bm{b}_i$ (media condicional o específica de
	la i-ésima unidad)
	\item $E(\bm{Y}_i) = \bm{X}_i\bm{\beta}$ (media marginal)
	\item $Cov(\bm{Y}_i/\bm{b}_i) = \bm{R}_i$ (variancia condicional)
	\item $Cov(\bm{Y}_i) = \bm{Z}_i \bm{D}_i \bm{Z}'_i + \bm{R}_i = \bm{\varSigma}_i$ (variancia marginal)
\end{itemize}

Generalmente, la matriz $\bm{D}$ adopta una estructura de covariancia
arbitraria, mientras que la matriz $\bm{R}_i$ adopta otra estructura que modela
apropiadamente la variabilidad intra individuo.

\subsection{Estimación de los parámetros del modelo}

Bajo el supuesto de que $\bm{\varepsilon}_i$ y $\bm{b}_i$ se distribuyen normalmente se
pueden usar métodos de estimación basados en la teoría de máxima verosimilitud,
cuya idea es asignar a los parámetros el valor más probable en base a los datos
que fueron observados. Se usarán para estimar los parámetros de la parte media
y los de las estructuras de covariancia los métodos de máxima verosimilitud
(ML) y máxima verosimilitud restringida (REML) respectivamente

\subsubsection{Método de máxima verosimilitud (ML)}

Bajo el supuesto de que $\bm{Y}_i \sim N_n(\bm{X}_i \bm{\beta},
\bm{\varSigma}_i)$ y las $\bm{Y}_i$
son independientes entre sí, se obtiene la siguiente función de
log-verosimilitud:

\begin{equation}
\label{ML}
	l = -\frac{1}{2} \sum_{i=1}^{N}n \ln(2\pi) - \frac{1}{2}\ln|\bm{\varSigma}_i| -
	\frac{1}{2} \sum_{i=1}^{N} [(\bm{Y}_i - \bm{X}_i\bm{\beta})'
	\bm{\varSigma}_i^{-1} (\bm{Y}_i - \bm{X}_i\bm{\beta})]
\end{equation}

Siendo $\bm{\varSigma}_i$ función del vector $\bm{\theta}$ que contiene los
parámetros de covariancia.

Los estimadores de $\bm{\beta}$ y $\bm{\theta}$ son los valores que maximizan
esta expresión. Cuando $\bm{\theta}$ es desconocido (lo que generalmente sucede)
se obtiene una ecuación no lineal, por lo que no se puede obtener una expresión
explícita de $\hat{\bm{\theta}}$. Para encontrar su solución se recurre a
métodos numéricos. El estimador del vector $\bm{\beta}$ resulta:

\[ \hat{\bm{\beta}} = (\sum_{i=1}^{N} \bm{X}_i'\hat{\bm{\varSigma}_i}^{-1}\bm{X}_i)^{-1}
\sum_{i=1}^{N} \bm{X}'_i\hat{\bm{\varSigma}_i}^{-1}\bm{Y}_i \]

El estimador $\hat{\bm{\beta}}$ resulta insesgado de $\bm{\beta}$. Cuando
$\bm{\theta}$ es desconocido no se puede calcular de manera exacta la matriz
de covariancias de $\hat{\bm{\beta}}$. Si el número de unidades es grande se
puede demostrar que asintóticamente (Fitzmaurice et al., 2004):

\[ \hat{\bm{\beta}} \sim N_p(\bm{\beta}, \bm{V}_{\bm{\beta}}) \quad donde \quad \bm{V}_{\bm{\beta}} =
(\sum_{i=1}^{N} \bm{X}'_i\hat{\bm{\varSigma}_i}^{-1}\bm{X}_i)^{-1} \]

\subsubsection{Método de máxima verosimilitud restringida (REML)}

El inconveniente que posee el método de ML es que los parámetros de covariancia
resultan sesgados. Es decir, a pesar de que $\bm{\hat{\beta}}$ es un estimador
insesgado de $\bm{\beta}$, no pasa lo mismo con $\bm{\hat{\theta}}$. Si el
tamaño de muestra es chico, los parámetros que representan las variancias van a
ser demasiado pequeños, dando así una visión muy optimista de la variabilidad de
las mediciones, es decir, se subestiman los parámetros de covariancia. El sesgo
se debe a que en la estimación ML de $\bm{\theta}$ no se tiene en cuenta que
$\bm{\beta}$ es estimado a partir de los datos.

Distintos autores proponen el método de REML para estimar los parámetros del
modelo. Este método es una modificación del método de máxima verosimilitud, en
el que la parte de los datos usada para estimar $\bm{\beta}$ está separada de
aquella usada para estimar los parámetros de $\bm{\varSigma}_i$. La función de
log-verosimilitud restringida que se propone es:

\begin{equation}
\label{REML}
	\bm{l}^* = -\frac{1}{2} \sum_{i=1}^{N}n \ln(2\pi) - \frac{1}{2}\ln|\bm{\varSigma}_i| -
	\frac{1}{2} \sum_{i=1}^{N} [(\bm{Y}_i - \bm{X}_i\bm{\beta})'
	\bm{\varSigma}_i^{-1} (\bm{Y}_i - \bm{X}_i\bm{\beta})] -
	- \frac{1}{2} \ln |\sum_{i=1}^{N} \bm{X}'_i \hat{\bm{\varSigma}_i^{-1} \bm{X}_i}|
\end{equation}

Maximizando esta funcion con respecto a $\bm{\beta}$ y $\bm{\theta}$ se obtiene:

\[ \hat{\bm{\beta}} = (\sum_{i=1}^{N} \bm{X}'_i \hat{\bm{\varSigma}}_i^{-1} \bm{X}_i)^{-1}
\sum_{i=1}^{N} \bm{X}'_i \hat{\bm{\varSigma}}_i^{-1} \bm{Y}_i\]

Donde $\hat{\bm{\varSigma}}_i$ es el estimador REML de ${\bm{\varSigma}_i}$ (Fitzmaurice et al., 2004).

\subsubsection{Problemas con la estimación}

Pepe y Anderson (1994) mostraron que las ecuaciones \ref{ML} y \ref{REML} llegan
a cero solo si se cumple con el supuesto de independencia condicional:

\begin{equation}
\label{estimation_issue}
	E[Y_{ij} | X_{ij}] = E[Y_{ij} | X_{ij}, j = 1, ..., n]
\end{equation}

Con las CNVT, esta suposición se mantiene necesariamente ya que $X_{ij} =
X_{ik}$ para todas las ocasiones $k \neq j$. Con las CVT estocásticas, que se
fijan por diseño del estudio (por ejemplo, indicador de grupo de tratamiento en
una prueba cruzada), la suposición también se cumple ya que los valores de las
covariables en cualquier ocasión se determinan a priori por diseño del estudio y
de manera completamente no relacionado con la respuesta longitudinal. Sin
embargo, cuando una covariable es variable en el tiempo no estocástica, puede
que no necesariamente se mantenga.

En general, cuando (\ref{estimation_issue}) no se cumple, los valores
precedentes y/o posteriores de la CVT confunden la relación entre $Y_{ij}$ y
$X_{ij}$, esto puede llevar a estimaciones sesgadas de los parámetros del
modelo.

Frente a este escenario, Pepe y Anderson (1994) recomendaron plantear el modelo
longitudinal marginal y realizar las estimaciones mediante GEE (ecuaciones de
estimación generalizadas) con estructura de correlación independiente ya que
este es siempre consistente. La estructura de correlación independiente
generalmente tiene una alta eficiencia para la estimación de los coeficientes
asociados a CNVT. Sin embargo, para las CVT, Fitzmaurice (1995) muestra que esta
estructura puede resultar en una pérdida sustancial de eficiencia para la
estimación de los coeficientes asociados a las CVT y proporciona un ejemplo en
el que la elección de dicha estructura tiene una eficiencia del 60\% en relación
con la estructura de correlación verdadera.

Luego, Lai y Small (2007) y Lalonde (2014) definieron 4 nuevos tipos de CVT:

\begin{itemize}
	\item Tipo I: una CVT es de Tipo I si no hay relación entre la
	CVT y la variable respuesta en diferentes ocasiones. Las variables que
	involucran cambios predecibles a lo largo del tiempo, como la edad o el
	tiempo de observación, generalmente se tratan como Tipo I.
	\item Tipo II: una CVT es de Tipo II si no está asociada a valores
	anteriores de la variable respuesta, pero la variable respuesta si puede
	estar asociada a valores previos de la CVT. Un ejemplo de este tipo de CVT
	puede ser la medicación para la presión arterial con respecto a la variable
	respuesta presión arterial, ya que valores acumulados de la medicación en
	el tiempo se espera que tengan un impacto en la presión arterial en
	cualquier ocasión.
	\item Tipo III: para éste tipo, no hay suposición de independencia entre la
	respuesta y los valores de la CVT en diferentes ocasiones. Por lo tanto, una
	CVT de Tipo III puede implicar un ciclo de feedback entre la CVT y la
	respuesta, en el que los valores de la covariable pueden verse afectados por
	los valores anteriores de la respuesta. Un ejemplo de este tipo es la CVT
	medicación para la presión arterial con la respuesta infarto de miocardio.
	Mientras que es esperado que la medicación impacte en la probabilidad de
	infarto, un evento de infarto puede provocar un cambio en la medicación para
	la presión arterial.
	\item Tipo IV: para una CVT de Tipo IV, la covariable puede estar asociada a
	valores previos de la variable respuesta, pero la respuesta no está asociada
	a valores previos de la CVT. Un ejemplo es la CVT presión arterial con la
	variable respuesta peso. Si bien existe una asociación entre el peso y la
	presión arterial, la dirección del efecto parece ser que el peso afecta la
	presión arterial, pero es poco probable que ocurra lo contrario.
\end{itemize}

Además, propusieron utilizar el ``Método generalizado de los momentos'' (GMM)
(Hansen, 1982). Éste método puede ser utilizado para tratar a cada CVT de manera
diferente, dependiendo del tipo de cada una, y evita problemas con la estimación
de ecuaciones construidas a partir de componentes no independientes.

En conclusión, si la CVT es exógena puede introducirse tanto a un modelo
estimado con GMM como a un modelo lineal mixto de manera tradicional. Sin
embargo, si la CVT es endógena, no puede introducirse al modelo lineal mixto y
debe definirse uno de sus 4 tipos para utilizar el GMM.

\newpage

%%%%%%%%%%%%%%%%%%%%%%%%%%%%%%%%%%%%%%%%%%%%%
% REESCRIBIR TODA ESTA SECCION (VER DIGGLE) %
%%%%%%%%%%%%%%%%%%%%%%%%%%%%%%%%%%%%%%%%%%%%%

\section{Formas de introducir una CVT al modelo}

Si al evaluar el tipo de la CVT de la manera vista en
\ref{seccion_de_exogeneidad} resulta ser exógena, se puede introducir en el
modelo lineal mixto sin consideraciones adicionales. Esto se debe a que no habrá
problemas con la estimación de los parámetros, ya que se cumple el supuesto de
independencia condicional.

A continuación, se presentan distintas maneras de introducir la CVT exógena al
modelo lineal mixto. De manera de ejemplo, se tomará un caso en el que la variable
respuesta $Y_{ij}$ y la CVT $X_{ij}$ son la tensión arterial y el IMC,
respectivamente, del paciente $i$ en la ocasión $j$.

\subsection{Convertirla en CNVT}

Una solución rápida al problema de las CVT, sea exógena o endógena, es
transformarla en una CNVT, esto se puede lograr resumiendo la información de la
misma mediante alguna función como el promedio de los valores de cada individuo
y dejarlo fijo a través del tiempo. También podria usarse su valor máximo,
mínimo o cualquier transformación que resulte de interés en el estudio. El
problema de este enfoque es que se pierde información, dado que se usa una
covariable más simple que no refleja la relación dinámica entre la covariable y
la respuesta en el tiempo.

En el ejemplo mencionado anteriormente, se podría calcular el IMC promedio de
cada uno de los individuos y el modelo resultaría:

\[ Y_{ij} = \beta_0 + b_{0i} + \beta_1 \overline{X}_i + \beta_2 t_j + b_{1i} t_j + \varepsilon_{ij} \]

El coeficiente $\beta_1$ se interpreta como el cambio en la tensión arterial por
cada cambio unitario en el IMC promedio del individuo.

\subsection{Covariable variable en el tiempo}

Dado que la CVT es exógena, se puede incorporar al modelo sin ninguna
transformación. El modelo resultante es,

\[ Y_{ij} = \beta_0 + b_{0i} + \beta_1 X_{ij} + \beta_2 t_j + b_{1i} t_j + \varepsilon_{ij} \]

Como se mencionó anteriormente, las CVT contienen efecto intra-unidad y
entre-unidad, sin embargo el modelo lineal mixto va a tener solo un coeficiente
$\beta_1$. Si bien el modelo es válido, para obtener una interpretación clara
del efecto de la CVT se necesita dividir su efecto en dos coeficientes, como se
explicará en la sección \ref{Dividiendo efecto entre-unidad y efecto
intra-unidad} (Hoffman, 2015).

%%%% LESSA HOFFMAN PAGINA 345

\subsection{Covariable rezagada}

En algunas aplicaciones hay justificacion previa para considerar la covariable
en el rezago $k$ momentos antes de la medición de la respuesta. Por ejemplo, el
efecto del IMC sobre la tensión arterial probablemente no sea inmediato, por lo
que podría interesarnos su valor en la ocasión anterior ($k=1$). Lo más común es
que se desconozca el valor $k$ apropiado y se consideren varias opciones
diferentes. El modelo lineal mixto se definiría de la siguiente manera:

\[ Y_{ij} = \beta_0 + b_{0i} + \beta_{1k} X_{ij-k} + \beta_2 t_j + b_{1i} t_j + \varepsilon_{ij} \]

En este modelo, el coeficiente $\beta_{1k}$ depende explícitamente de la
elección del rezago $k$.

\subsection{Funcion de las covariables rezagadas}

Una alternativa cuando se quiere utilizar la información de las covariables
rezagadas es resumir a través de una función la información de éstas en una sola
covariable. Un ejemplo puede ser el valor promedio o acumulado hasta la ocasión
actual. Sin embargo, la elección de está funcion dependerá del tipo de problema
a analizar. Cabe destacar que, al igual que con toda medida resumen, al usar
este tipo de covariables se pierde parte de la información. En el ejemplo
mencionado, podría ser de interés calcular el IMC promedio hasta la ocasión $j$,
resultando el modelo:

\[ Y_{ij} = \beta_0 + b_{0i} + \beta_1 \overline{X}_{ij} + \beta_2 t_j + b_{1i} t_j + \varepsilon_{ij} \]

Donde $\overline{X}_{ij}$ es el IMC promedio calculado hasta la ocasión $j$

\subsection{Dividiendo efecto entre-unidad y efecto intra-unidad}
\label{Dividiendo efecto entre-unidad y efecto intra-unidad}

Otra forma de incorporar la CVT es dividiendo el efecto en dos componentes que
reflejen la variación intra-unidad y la variación entre-unidad. Entonces, el
coeficiente del modelo que representa a la covariable se puede descomponer en
dos:

\[ 
	Y_{ij} = \beta_0 + b_{0i} + \beta_W (X_{ij} - \overline{X}_i) + \beta_B
	\overline{X}_i + \beta_2 t_j + b_{1i} t_j + \varepsilon_{ij}
\]

Donde, $\overline{X}_i$ representa el promedio de todos los valores observados
en el tiempo de la CVT para la unidad $i$, $\beta_W$ representa el cambio
esperado en la media de la variable respuesta asociado con cambios de la CVT
dentro de la unidad y $\beta_B$ representa el cambio esperado en la media de
la variable respuesta asociado con cambios de la CVT entre unidades

En el ejemplo del IMC y la tensión arterial, $\beta_W$ se interpreta como el
cambio en la tensión arterial por cada diferencia de una unidad en el IMC de la
ocasión $j$ comparado con el IMC promedio de la persona, manteniendo constante
el IMC de las ocasiones restantes. Por otro lado, $\beta_B$ se interpreta como
el cambio en la tensión arterial por cada cambio de una unidad en el IMC
promedio de la unidad, controlando por el valor del IMC de la ocasión $j$.

Estos dos efectos permiten medir entonces si la tensión arterial aumenta cuando
una persona tiene un IMC mayor que el promedio de la población, y también
permiten medir el cambio en la tensión arterial si dicha persona tiene IMC mayor
que su promedio habitual.

\newpage

\section{Aplicación}

A partir del programa de atención y control de pacientes hipertensos iniciado en
el año 2014 en Rosario se obtienen datos de \npatients{} pacientes de entre 30 y
86 años, con una edad media de 58.84 y desvío estándar de 9.87 años, de los
cuales un 49.28\% son hombres. A todos estos pacientes se les indicó un
tratamiento antihipertensivo al incio del seguimiento (visita basal). Durante 7
meses se agendaron visitas mensuales en donde se registraron, entre otras
características, el valor de la TAS y la adherencia al tratamiento. Esta última
variable surge de la evaluación del cuestionario Morisky (Morisky et al., 1986)
que arroja como resultado: adhiere al tratamiento, no adhiere al tratamiento. Al
ser evaluada en todas las visitas mensuales, esta variable dicotómica es una CVT
que captura como fue la adherencia durante el periodo desde la visita previa
hasta la visita actual. Como para la visita basal no se cuenta con información
de adherencia, se toman los datos desde el primer mes de tratamiento.

Uno de los objetivos que persiguió este estudio fue evaluar si la adherencia
influye en los valores de la TAS a lo largo del seguimiento.

Para dar respuesta a este interrogante se propuso ajustar un modelo longitudinal
de efectos mixtos considerando a la TAS como variable respuesta y la adherencia,
sexo y edad como variables explicativas.

Para todas las decisiones de esta sección se utilizará un nivel de significación
del 5\%.

\subsection{Nomenclatura}
\label{variables}

A continuación se describen las variables originales que se encuentran en el
dataset y sus distintas transformaciones transformaciones.

Siendo $ i = 1, ..., \npatients{}$ y $j = 1, ..., 7$ se obtienen:

\begin{itemize}
	\item $TAS_{ij}$: tension arterial sistólica (mmHg) del paciente $i$ en el
	mes $j$.
	\item $\overline{TAS}_{i}$: tension arterial sistólica (mmHg) promedio del
	paciente $i$ a lo largo del seguimiento ($\sum_{k=0}^n \frac{TAS_{ik}}{n}$).
	\item $\overline{TAS}_{ij}$: tension arterial sistólica (mmHg) promedio del
	paciente $i$ hasta el mes $j$ ($\sum_{k=0}^j \frac{TAS_{ik}}{j}$).
	\item $sexo_i$: sexo del paciente $i$ medido en la ocasión basal (mes 0).
	\item $edad_i$: edad del paciente $i$ medido en la ocasión basal (mes 0).
	\item $mes_j$: meses transcurridos desde el inicio del tratamiento hasta la
	ocasión $j$.
	\item $adherencia_{ij}$: adherencia al tratamiento del paciente $i$ en el
	mes $j$ (variable dicotómica: =1 si adhiere, =0 si no adhiere).
	\item $\overline{adherencia}_i$: proporción de visitas en las que el
	paciente $i$ adhirió al tratamiento a lo largo del seguimiento
	($\sum_{k=0}^n \frac{adherencia_{ik}}{n}$).
	\item $\overline{adherencia}_{ij}$: proporción de visitas en las que el
	paciente $i$ adhiere al tratamiento hasta el mes $j$ ($\sum_{k=0}^j
	\frac{adherencia_{ik}}{j}$).
	\item $adherencia\ perfecta_i$: variable indicadora, $=1$ si el paciente
	$i$ adhirió al tratamiento todos los meses, $=0$ en otro caso.
\end{itemize}

\subsection{Análisis descriptivo}

En esta sección se presentaran diversos gráficos con el fin de describir la
población en estudio.

En la figura \ref{TAS_vs_tpo} se puede observar el grupo de pacientes inició el
tratamiento con una TAS promedio de casi 133 mmHg, la cual fue disminuyendo
levemente de manera aproximadamente lineal hasta promedio de poco más de 130
mmHg al final del tratamiento.

\begin{figure}[H]
	\centering
	\includegraphics[scale=0.5]{img/TAS_vs_tpo.png}
	\caption{Evolución de la TAS promedio a lo largo del tratamiento}
	\label{TAS_vs_tpo}
\end{figure}

Otro gráfico que resulta de interés es observar la evolución de la TAS en cada
mes sobre cada grupo de las covariables fijas, el resultado es presentado en la
figura \ref{TAS_with_covs}. Todos los perfiles presentan (en general) una
pendiente decreciente de la TAS promedio a través del tiempo, siendo esta menor
en cuanto a la ordenada al origen en el grupo de pacientes de mayor edad y de
sexo femenino. La covariable edad es continua, pero a fines de poder observar
los perfiles promedio de los pacientes se dividió en dos grupos según menores o
mayores o iguales 59 años (mediana de edad de los pacientes).

\begin{figure}[H]
	\centering
	\includegraphics[scale=0.4]{img/TAS_vs_tpo_with_covs.png}
	\caption{Evolución de la TAS promedio a lo largo del tratamiento según sexo y edad}
	\label{TAS_with_covs}
\end{figure}

Cuando se cuenta con una CVT, cada individuo puede presentar distintos valores
de la covariable en cada ocasión. Por lo tanto, para poder visualizar el efecto
de la adherencia sobre la TAS según perfiles promedio, es necesario convertir la
CVT en una CNVT para que cada paciente pertenezca únicamente a un solo grupo.
Para eso, se dividieron distintos perfiles de pacientes según la cantidad de
meses que adhirieron al tratamiento, formando los grupos: 3 meses o menos, entre
4 y 6 meses o todos los meses. Se puede observar que para el grupo de pacientes
que adhirieron al tratamiento 3 meses o menos la TAS presenta una pendiente
creciente a lo largo del estudio. Para los otros 2 grupos, la TAS disminuye a lo
largo del tratamiento, siendo menor para el grupo de pacientes que adhirieron la
totalidad de los meses.

\begin{figure}[H]
	\centering
	\includegraphics[scale=0.5]{img/TAS_vs_tpo_with_adherencia.png}
	\caption{TAS a través del tiempo según perfiles de adherencia al tratamiento}
	\label{TAS_with_adh}
\end{figure}

\subsection{Evaluación de la exogeneidad}

Para evaluar la exogeneidad de la variable adherencia es necesario ajustar un
modelo para cada ocasión en el que se considere a la variable adherencia como
variable respuesta y como variables explicativas a la TAS y la adherencia en
ocasiones previas. Cómo la CVT adherencia es una variable dicotómica, se ajustan
modelos de regresión logística.

Para ajustar estos modelos se utilizarán como covariables a la adherencia y la
TAS en el mes anterior, ya que se asume que las mediciones más cercanas entre sí
están más correlacionadas y tambien se utilizarán la adherencia y la TAS
promedio desde el inicio hasta 2 meses antes, de esta manera se puede utilizar
toda la información del estudio. El modelo resulta:

\[ 
	logit(adherencia_{ij}) = \beta_0 + \beta_1\ adherencia_{ij-1} + \beta_2\ TAS_{ij-1}
	+ \beta_3\ \overline{adherencia}_{ij-2} + \beta_4\ \overline{TAS}_{ij-2} + \varepsilon_{ij}
\]

Siendo

\[ \varepsilon_{ij} \sim N(0, Var(\varepsilon_{ij})) \]

En la tabla \ref{exog_table} se presentan los valores de los coeficientes para
cada covariable y en paréntesis el p-value asociado a cada uno. Como se puede
notar, en ninguna ocasión la adherencia depende de valores anteriores de la TAS,
por lo tanto puede considerarse como una covariable exógena.

\begin{table}[H]
	\centering
	\caption{Estimación de coeficientes de los modelos logit y sus respectivos p-value}
	\label{exog_table}
	\begin{tabular}{*{5}{|c}|}
		\hline
		mes\ (j) & $adherencia_{ij-1}$ & $\overline{adherencia}_{ij-2}$ & $TAS_{ij-1}$ &
		$\overline{TAS}_{ij-2}$ \\
		\hline
		\hline
		$2$ & $1.9302\ (<0.001)$ & $-$ & $0.0057\ (0.45)$ & $-$ \\
		$3$ & $2.3047\ (<0.001)$ & $0.5683\ (0.044)$ & $-0.0088\ (0.343)$ &
		$0.0075\ (0.419)$ \\
		$4$ & $1.9689\ (<0.001)$ & $1.0734\ (0.002)$ & $0.0138\ (0.17)$ &
		$-0.017\ (0.138)$ \\
		$5$ & $2.2945\ (<0.001)$ & $1.0617\ (0.007)$ & $0.0092\ (0.441)$ &
		$-0.0141\ (0.307)$ \\
		$6$ & $2.2741\ (<0.001)$ & $1.0698\ (0.015)$ & $-0.0008\ (0.938)$ &
		$<0.0001\ (0.996)$ \\
		$7$ & $2.5812\ (<0.001)$ & $1.4609\ (0.003)$ & $-0.0005\ (0.966)$ &
		$-0.0072\ (0.678)$ \\
		\hline
	\end{tabular}
\end{table}

\subsection{Modelo lineal mixto propuesto}

Mediante la inspección del semivariograma muestral (figura \ref{semivariogram})
y pruebas de hipótesis para efectos aleatorios (ver Anexo
\ref{eleccion_efectos_aleatorios}) se decidió plantear un modelo de efectos
mixtos con ordenada y pendiente aleatoria. De esta manera, se incorpora al
modelo la correlación serial y el efecto entre pacientes. La CVT se incorpora en
su forma original ya que se evaluó su exogeneidad. El modelo resulta:

% \subsubsection{Modelo propuesto para la media}

% En una primera instancia, se propone un modelo lineal de efectos mixtos sin
% considerar a la CVT, ya que primero debe definirse la estructura de covariancia
% y probar su exogeneidad.

% \[
% 	Y_{ij} = \beta_0 + \beta_1 sexo_i + \beta_2 edad_i +
% 	\beta_3 mes_j + \varepsilon_{ij}
% \]

% Donde 

% \[
% 	\bm{\varepsilon}_i = \begin{pmatrix} \varepsilon_{i1} \\ \vdots \\ \varepsilon_{in} \end{pmatrix} \sim N_{n}(0, \bm{R}_i)
% \]


% Siendo $\bm{R}_i$ la matriz de variancias y covariancias del vector $\bm{\varepsilon}_i$.

% \subsubsection{Modelo propuesto para la covariancia}

% En la figura \ref{semivariogram} puede notarse que la mayor parte de la
% variabilidad total está compuesta por el error de medición dado que la curva no
% inicia en el cero. Al no haber una pendiente muy pronunciada, la correlación
% serial también puede considerarse pequeña. Por último, como la curva no llega a
% la variancia total, esto nos indica que debe explicarse la variancia entre
% individuos agregando una ordenada aleatoria.

% %%%%%%%%%%%%%%%%%%%%%%%%%%%%
% % AGREGAR PROCESO AL ANEXO %
% %%%%%%%%%%%%%%%%%%%%%%%%%%%%

% Además, se realizaron tests de hipótesis para probar la significación de los
% efectos aleatorios para la ordenada y la pendiente, resultando ambos
% significativos (ver Anexo \ref{eleccion_efectos_aleatorios}). Por lo que la
% estructura de covariancia elegida para el modelo es una estructura independiente
% con ordenada y pendiente aleatoria.

% El modelo resultante se escribe:

\[
	Y_{ij} = \beta_0 + b_{0i} + \beta_1\ sexo_i + \beta_2\ edad_i + \beta_3\ adherencia_{ij}
	+ \beta_4\ mes_j + \beta_5\ mes_j\ adherencia_{ij} + mes_j\ b_{1i} + \varepsilon_{ij}
\]

Se supone que $\bm{\varepsilon}_i$ y $\bm{b}_i$ son independientes.

\[ 
	\bm{\varepsilon}_i = \begin{pmatrix} \varepsilon_{i1} \\ \vdots \\ \varepsilon_{in} \end{pmatrix} \sim N_{n}(0, \bm{R}_i)
	\quad
	\bm{b}_i = \begin{pmatrix} b_{0i} \\ b_{1i} \end{pmatrix} \sim N_k(0, \bm{D})
\]

Donde $\bm{D}$ y $\bm{R}_i$ son las matrices de variancias y covariancias de los
vectores $\bm{b}_i$ y $\bm{\varepsilon}_i$ respectivamente.

\begin{figure}[H]
	\centering
	\includegraphics[scale=0.4]{img/semivariogram.png}
	\caption{Semivariograma muestral}
	\label{semivariogram}
\end{figure}

Los coeficientes del modelo estimado se presentan en la tabla
\ref{modelo_propuesto}.

\begin{table}[H]
	\centering
	\caption{Modelo propuesto}
	\label{modelo_propuesto}
	\begin{tabular}{*{5}{|c}|}
		\hline
		\multicolumn{3}{|c}{Log-Likelihood} & \multicolumn{2}{|c|}{-15390.73} \\
		\multicolumn{3}{|c}{AIC} & \multicolumn{2}{|c|}{30801.47} \\
		\multicolumn{3}{|c}{BIC} & \multicolumn{2}{|c|}{30864.21} \\
		\hline
		Covariable				 & Coef.   & Std. Err. & z      & $P<|z|$  \\
		\hline
		$intercepto$             & 122,270 & 2,502     & 48,874 & $<0.001$ \\
		$sexo_i$                 & 3,760   & 0,747     & 5,036  & $<0.001$ \\
		$edad_i$                 & 0,167   & 0,038     & 4,357  & $<0.001$ \\
		$adherencia_{ij}$        & -1,524  & 1,143     & -1,333 & $0.183$  \\
		$mes_j$                  & 0,371   & 0,240     & 1,547  & $0.122$  \\
		$mes_j\ adherencia_{ij}$ & -0,792  & 0,262     & 3,018  & $0.003$  \\
		\hline
	\end{tabular}
\end{table}

Como se mencionó anteriormente, la CVT adherencia contiene ambos efectos entre e
intra paciente, por lo tanto su coeficiente no será interpretable y será dificil
sacar conclusiones sobre como afecta la adherencia a la TAS. En la siguiente
sección se verán distintas formas de incorporar esta CVT al modelo, las cuales
algunas si serán interpretables.

\subsection{Incorporación de la CVT}

Hay más de una manera de incorporar una CVT exógena a un modelo lineal mixto, en
esta sección compararemos algunas de ellas.

\subsubsection{Incorporación de covariable fija}

Hay diversas formas de convertir una CVT en CNVT, en este apartado se mostrarán
2 que resultan de interés para el estudio. Es importante destacar que éstas
transformaciones se pueden utilizar para incorporar CVT tanto exógenas como
endógenas.

Una de las transformaciones que puede aplicarse sobre la covariable adherencia
es convertirla en una variable dicotómica fija, cuyo valor es 1 si el paciente
adhirió en todo el tratamiento y 0 en otro caso. En base a los resultados
presentados en la tabla \ref{modelo_1_tabla}, controlando por el resto de las
variables, los pacientes que adhieren en la totalidad del tratamiento, inician
el tratamiento con una TAS promedio menor en 2,763 unidades, y luego ésta
continúa descendiendo en 0,272 unidades por mes.

\begin{multline}
	\label{modelo_1}
	Y_{ij} = \beta_0 + b_{0i} + \beta_1\ sexo_i + \beta_2\ edad_i + \beta_3\ adherencia\ perfecta_i \\
	+ \beta_4\ mes_j + \beta_5\ mes_j\ adherencia\ perfecta_i + b_{1i}\ mes_j + \varepsilon_{ij}
\end{multline}

\begin{table}[H]
	\centering
	\caption{Modelo 1: Incorporación adherencia perfecta}
	\label{modelo_1_tabla}
	\begin{tabular}{*{5}{|c}|}
		\hline
		\multicolumn{3}{|c}{Log-Likelihood} & \multicolumn{2}{|c|}{-15415.08} \\
		\multicolumn{3}{|c}{AIC} & \multicolumn{2}{|c|}{30850.16} \\
		\multicolumn{3}{|c}{BIC} & \multicolumn{2}{|c|}{30912.9} \\
		\hline
		Covariable					    & Coef.   & Std. Err. & z      & $P<|z|$  \\
		\hline
		$intercept$                     & 121,926 & 2,376     & 51,320 & $<0.001$ \\
		$sexo_i$                        & 3,801   & 0,740     & 5,133  & $<0.001$ \\
		$edad_i$                        & 0,176   & 0,038     & 4,590  & $<0.001$ \\
	    $adherencia\ perfecta_i$        & -2,491  & 1,176     & -2,119 & $0.034$  \\
		$mes_j$                         & -0,197  & 0,149     & -1,322 & $0.186$  \\
		$mes_j\ adherencia\ perfecta_i$ & -0,272  & 0,212     & -1,280 & $0.201$  \\
		\hline
	\end{tabular}
\end{table}

Otra manera de incorporar la covariable fija, conservando más información, es
usar la proporción de adherencia total al final del estudio. Es decir, si de los
7 meses el paciente adhiere en solo 5, el valor que se le asignará es
$\frac{5}{7}$ ($\approx 0.71$). Los coeficientes presentados en la tabla
\ref{modelo_2_tabla} indican que, controlando por el resto de las variables, los
pacientes presentan una disminución promedio en la TAS de 2,838 en proporción a
la adherencia total al inicio del tratamiento y luego disminuirá en 1,010
proporcional a la adherencia total por mes. Por ejemplo, un paciente que adhiere
sólo al 50\% del tratamiento, tendrá una disminución promedio en su TAS de 1.419
($2,838\ *\ 0.5$) al inicio del tratamiento, y luego una disminución de 0,505
por mes ($1,010\ *\ 0.5$).

\begin{multline}
	\label{modelo_2}
	Y_{ij} = \beta_0 + b_{0i} + \beta_1\ sexo_i + \beta_2\ edad_i + \beta_3\ \overline{adherencia}_i \\
	+ \beta_4\ mes_j + \beta_5\ mes_j\ \overline{adherencia}_i + b_{1i}\ mes_j + \varepsilon_{ij}
\end{multline}

\begin{table}[H]
	\centering
	\caption{Modelo 2: incorporación adherencia total}
	\label{modelo_2_tabla}
	\begin{tabular}{*{5}{|c}|}
		\hline
		\multicolumn{3}{|c}{Log-Likelihood} & \multicolumn{2}{|c|}{15417.66} \\
		\multicolumn{3}{|c}{AIC} & \multicolumn{2}{|c|}{30855.32} \\
		\multicolumn{3}{|c}{BIC} & \multicolumn{2}{|c|}{30918.06} \\
		\hline
		Covariable 			 			 & Coef.   & Std. Err. & z 	    & $P<|z|$  \\
		\hline
		$intercept$                      & 122,419 & 3,036     & 40,318 & $<0.001$ \\
		$sexo_i$                         & 3,732   & 0,747     & 4,995  & $<0.001$ \\
		$edad_i$                         & 0,172   & 0,039     & 4,469  & $<0.001$ \\
		$\overline{adherencia}_i$        & -1,828  & 2,158     & -0,736 & $0.462$  \\
		$mes_j$                          & 0,498   & 0,380     & 1,308  & $0.191$  \\
		$mes_j\ \overline{adherencia}_i$ & -1,010  & 0,445     & -2,270 & $0.023$  \\
		\hline
	\end{tabular}
\end{table}

\subsubsection{Incorporación como CVT}

Se proponen otras dos maneras de incorporar la CVT al modelo, la adherencia en
el mes anterior y la adherencia promedio hasta el mes actual, ya que podria
pensarse que el tratamiento no es de efecto inmediato y la TAS podria estar
influenciada por la adherencia en meses anteriores.

Los resultados de ajustar el modelo utilizando la adherencia en el mes anterior
se presentan en la tabla \ref{modelo_3_tabla}.

\begin{multline}
	\label{modelo_3}
	Y_{ij} = \beta_0 + b_{0i} + \beta_1\ sexo_i + \beta_2\ edad_i + \beta_3\ adherencia_{ij-1} \\
	+ \beta_4\ mes_j + \beta_5\ mes_j\ adherencia_{ij-1} + b_{1i}\ mes_j + \varepsilon_{ij}
\end{multline}

\begin{table}[H]
	\centering
	\caption{Modelo 3: incorporación adherencia en el mes anterior}
	\label{modelo_3_tabla}
	\begin{tabular}{*{5}{|c}|}
		\hline
		\multicolumn{3}{|c}{Log-Likelihood} & \multicolumn{2}{|c|}{-15422.50} \\
		\multicolumn{3}{|c}{AIC} & \multicolumn{2}{|c|}{30865.0} \\
		\multicolumn{3}{|c}{BIC} & \multicolumn{2}{|c|}{30927.74} \\
		\hline
		Covariable 		   		   & Coef.   & Std. Err. & z      & $P<|z|$    \\
		\hline
		$intercept$                & 121,125 & 2,388 	 & 50,714 & $<0.001$   \\
		$sexo_i$                   & 3,869   & 0,751 	 & 5,151  & $<0.001$   \\
		$edad_i$                   & 0,162   & 0,039 	 & 4,197  & $<0.001$   \\
		$adherencia_{ij-1}$        & 0,189   & 0,815 	 & 0,232  & $0.817$    \\
		$mes_j$                    & 0,064   & 0,179 	 & 0,359  & $0.720$    \\
		$mes_j\ adherencia_{ij-1}$ & -0,443  & 0,215 	 & -2,061 & $0.039$    \\
		\hline
	\end{tabular}
\end{table}

En la tabla \ref{modelo_4_tabla} se presentan los resultados de un modelo
ajustando utilizando la adherencia promedio en cada mes. Es decir, si en el mes
4 un paciente adhirio solo en 2 meses, su adherencia promedio será de
$\frac{2}{4}$ ($0.5$). Sin embargo, si adhiere en la ocasión 5, entonces la
adherencia promedio en ese mes será $\frac{3}{5}$ ($0.6$). 

\begin{multline}
	\label{modelo_4}
	Y_{ij} = \beta_0 + b_{0i} + \beta_1\ sexo_i + \beta_2\ edad_i + \beta_3\ overline{adherencia}_{ij} \\
	+ \beta_4\ mes_j + \beta_5\ mes_j\ \overline{adherencia}_{ij} + b_{1i}\ mes_j + \varepsilon_{ij}
\end{multline}

\begin{table}[H]
	\centering
	\caption{Modelo 4: incorporación adherencia acumulada}
	\label{modelo_4_tabla}
	\begin{tabular}{*{5}{|c}|}
		\hline
		\multicolumn{3}{|c}{Log-Likelihood} & \multicolumn{2}{|c|}{-15409.38} \\
		\multicolumn{3}{|c}{AIC} & \multicolumn{2}{|c|}{30838.77} \\
		\multicolumn{3}{|c}{BIC} & \multicolumn{2}{|c|}{30901.51} \\
		\hline
		Covariable 				 			& Coef.   & Std. Err. & z      & $P<|z|$ \\
		\hline
		$intercept$                	   		& 123,107 & 2,586 	  & 47,603 & $<0.001$ \\
		$sexo_i$                       		& 3,721   & 0,749 	  & 4,966  & $<0.001$ \\
		$edad_i$                       		& 0,177   & 0,039 	  & 4,574  & $<0.001$ \\
		$\overline{adherencia}_{ij}$   		& -3,247  & 1,421 	  & -2,285 & $0.022$ \\
		$mes_j$                        		& 0,382   & 0,306 	  & 1,248  & $0.212$ \\
		$mes_j\ \overline{adherencia}_{ij}$ & -0,835  & 0,355 	  & -2,348 & $0.019$ \\
		\hline
	\end{tabular}
\end{table}

Al igual que con el modelo propuesto en una primera instancia, estas CVT siguen
conteniendo ambos efectos entre e intra paciente, por lo que sus coeficientes no
pueden ser fácilmente interpretables.

\subsubsection{Incorporación dividiendo efecto entre e intra}

Cuando la CVT es dicotómica, con 0 indicando la ausencia del atributo y 1 la
presencia, entonces $\overline{X}_i$ es la proporción en la que una persona
presentó el valor codificado con 1 de dicha covariable, por lo que el método
presentado en la sección \ref{Dividiendo efecto entre-unidad y efecto
intra-unidad} resultará en valores extraños para $\beta_W$ si usamos $X_i -
\overline{X}_i$. En este caso, si paciente adhirio al tratamiento en el 50\% de
los meses, $\overline{X}_i$ tendrá un valor de 0.5 y entonces el término que
acompaña a $\beta_W$ será de -0.5 en los meses que el paciente no adhiera y 0.5
en los meses que si adhiera. En términos de la estimación del modelo esto no
genera ningún problema, pero será confuso en la interpretación de los
parámetros, dado que el parámetro $\beta_W$ estará siempre presente (nunca
estará acompañado de un 0). Por lo tanto, para evitar esto, el efecto
intra-unidad estará acompañado solo de $X_{ij}$

Los resultados pueden observarse en la tabla \ref{modelo_5_tabla}. Los
coeficientes -2.658 y 0,603 quieren decir que, manteniendo constante la
proporción de adherencia total, se espera que los pacientes que adhieren al
tratamiento presenten una disminución promedio en la TAS de 2.658 unidades al
inicio del tratamiento y que esta disminuya en promedio 0,603 los meses en los
que se adhiere al tratamiento. Por otro lado, los coeficientes -0,137 y -0,525
quieren decir que, después de controlar por la adherencia en ese mes, se espera
que la TAS al inicio del tratamiento sea mayor en 0.137 unidades relativo a la
proporción de adherencia total y luego disminuya, en promedio, 0,525 unidades en
proporcion a la adherencia total al tratamiento.

% Los resultados pueden observarse en la tabla \ref{modelo_5_tabla}. El
% coeficiente -1.261 quiere decir que, manteniendo constante la proporción de
% adherencia total, se espera que la TAS disminuya en promedio 1.261 los meses en
% los que se adhiere al tratamiento. Por otro lado, el coeficiente -0.246 quiere
% decir que, después de controlar por la adherencia en ese mes, se espera que la
% TAS disminuya en promedio en 0.246 relativo a la proporción de adherencia total.

\begin{multline}
	\label{modelo_5}
	Y_{ij} = \beta_0 + b_{0i} + \beta_1\ sexo_i + \beta_2\ edad_i + \beta_{3}\ adherencia_{ij} + \beta_4\ \overline{adherencia}_i \\
	+ \beta_5\ mes_j + \beta_6\ mes_j\ adherencia_{ij} + \beta_7\ mes_j\ \overline{adherencia}_i + b_{1i}\ mes_j + \varepsilon_{ij}
\end{multline}

\begin{table}[H]
	\centering
	\caption{Modelo 5: incorporación la adherencia dividiendo efecto entre e
	intra persona}
	\label{modelo_5_tabla}
	\begin{tabular}{*{5}{|c}|}
		\hline
		\multicolumn{3}{|c}{Log-Likelihood} & \multicolumn{2}{|c|}{-15389.83} \\
		\multicolumn{3}{|c}{AIC} & \multicolumn{2}{|c|}{30803.66} \\
		\multicolumn{3}{|c}{BIC} & \multicolumn{2}{|c|}{30878.95} \\
		\hline
		Covariable 			 			 & Coef.   & Std. Err. & z      & $P<|z|$  \\
		\hline
		$intercept$            			 & 121,987 & 3,032 	   & 40,235 & $<0.001$ \\
		$sexo_i$                   		 & 3,723   & 0,747 	   & 4,983  & $<0.001$ \\
		$edad_i$                   		 & 0,171   & 0,039 	   & 4,429  & $<0.001$ \\
		$adherencia_{ij}$            	 & -2,055  & 1,325 	   & -1,550 & $0.121$  \\
		$\overline{adherencia}_i$        & 0,662   & 2,784 	   & 0,238  & $0.812$  \\
		$mes_j$                    		 & 0,644   & 0,379 	   & 1,701  & $0.089$  \\
		$mes_j\ adherencia_{ij}$         & -0,603  & 0,314 	   & -1,919 & $0.055$  \\
		$mes_j\ \overline{adherencia}_i$ & -0,525  & 0,529 	   & -0,992 & $0.321$  \\
		\hline
	\end{tabular}
\end{table}

\subsubsection{Comparación de los métodos}

\begin{table}[H]
	\centering
	\caption{Coeficientes estimados con sus respectivos p-values y criterio de Akaike de cada modelo}
	\label{comparacion}
	\begin{tabular}{*{3}{|c}|}
		\hline
		Modelo 			 & AIC 		& BIC 	   \\
		\hline
		Modelo propuesto & 30801.47 & 30864.21 \\
		Modelo 1 		 & 30850.16 & 30912.9  \\
		Modelo 2 		 & 30855.32 & 30918.06 \\
		Modelo 3 		 & 30865.0  & 30927.74 \\
		Modelo 4 		 & 30838.77 & 30901.51 \\
		Modelo 5 		 & 30803.66 & 30878.95 \\
		\hline
	\end{tabular}
\end{table}

En la tabla \ref{comparacion} se puede observar que, basándose en el AIC y el
BIC, los modelos que mejor ajustan los datos son el modelo propuesto y el modelo
5, es decir, los modelos que incorporan la CVT en su manera natural sin
aplicarse ninguna transformación. El modelo 5 presenta valores un poco
superiores, esto se debe a que se agregan coeficientes que no son significativos
pero logran facilitar su interpretación.

\newpage

\section{Conclusiones}

Tradicionalmente, los modelos longitudinales fueron pensados para ser ajustados
sobre covariables fijas a través del tiempo, pero esto no es algo que suceda
siempre en la vida real.

En este informe se ha introducido una manera de categorizar a las covariables
variables en el tiempo, específicamente como \textit{exógenas} o
\textit{endógenas}, como así también un método para verificar esta clasificación
a través de ajustar disintos modelos individualmente para cada ocasión.

Cuando las variables son exógenas pueden añadirse al modelo de la manera
tradicional. Sin embargo, se han propuesto diversas transformaciones que pueden
ayudar tanto a ajustar de mejor manera los datos como a la interpretación de los
coeficientes.

También se mostró un ejemplo con un caso de estudio de pacientes de un programa
de atención y control de pacientes hipertensos. En base a estos datos se
ajustaron diferentes modelos con las distintas formas mencionadas de introducir
la CVT y se realizó una comparación de los mismos.

Como futuros pasos se propone estudiar más en profundidad la incorporación
de variables endógenas. Dado que muchas de las técnicas existentes hasta el
momento no están basadas en el ajuste de modelos lineales mixtos, quedan fuera
del alcance de esta tesina.

\newpage

\section{Anexo}

\subsection{Elección de efectos aleatorios}
\label{eleccion_efectos_aleatorios}

Para probar la siginificación de los efectos aleatorios se ajustaron 3 modelos,
uno con ordenada y pendiente aleatoria, otro solo con pendiente aleatoria y otro
solo con ordenada aleatoria. Los parametros de los 3 modelos fueron estimados
con el método de máxima verosimilitud restringida. El modelo completo es el
siguiente y sus resultados pueden observarse en la tabla \ref{modelo_both}.

\[
	Y_{ij} = \beta_0 + b_{0i} + \beta_1\ sexo_i + \beta_2\ edad_i + \beta_3\ adherencia_{ij}
	+ \beta_4\ mes_j + \beta_5\ mes_j\ adherencia_{ij} + b_{1i}\ mes_j + \varepsilon_{ij}
\]

Se supone que $\bm{\varepsilon}_i$ y $\bm{b}_i$ son independientes.

\[ 
	\bm{\varepsilon}_i = \begin{pmatrix} \varepsilon_{i1} \\ \vdots \\ \varepsilon_{in} \end{pmatrix} \sim N_{n}(0, \bm{R}_i)
	\quad
	\bm{b}_i = \begin{pmatrix} b_{0i} \\ b_{1i} \end{pmatrix} \sim N_k(0, \bm{D})
\]

Donde $\bm{D}$ y $\bm{R}_i$ son las matrices de variancias y covariancias de los
vectores $\bm{b}_i$ y $\bm{\varepsilon}_i$ respectivamente.

Siendo \[
	Var(\bm{b_i}) = Var\begin{pmatrix} b_{0i} \\ b_{1i} \end{pmatrix}
	= \bm{D} = \begin{pmatrix} D_{00} & D_{01} \\ D_{10} & D_{11} \end{pmatrix}
\]

\begin{table}[H]
	\centering
	\caption{Modelo con ordenada y pendiente aleatoria}
	\label{modelo_both}
	\begin{tabular}{*{5}{|c}|}
		\hline
		\multicolumn{3}{|c}{Log-Likelihood} & \multicolumn{2}{|c|}{-15393,93} \\
		\multicolumn{3}{|c}{Var($intercepto$)} & \multicolumn{2}{|c|}{112,78} \\
		\multicolumn{3}{|c}{Var($mes_j$)} & \multicolumn{2}{|c|}{2,16} \\
		\multicolumn{3}{|c}{Cov($Intercepto$, $mes_j$)} & \multicolumn{2}{|c|}{-10,6} \\
		\hline
		Covariable 				 & Coef.   & Std. Err. & z      & $P<|z|$  \\
		\hline
		$intercepto$             & 122,271 & 2,508     & 48,751 & $<0.001$ \\
		$sexo_i$                 & 3,760   & 0,749     & 5,022  & $<0.001$ \\
		$edad_i$                 & 0,167   & 0,039     & 4,344  & $<0.001$ \\
		$adherencia_{ij}$        & -1,525  & 1,144     & -1,333 & $0.183$  \\
		$mes_j$                  & 0,371   & 0,240     & 1,545  & $0.122$  \\
		$mes_j\ adherencia_{ij}$ & -0,792  & 0,263     & -3,015 & $0.003$  \\
		\hline
	\end{tabular}
\end{table}

Se ajusta un modelo reducido solo con pendiente aleatoria, obteniendose los
resultados presentados en \ref{modelo_time}:

\[
	Y_{ij} = \beta_0 + \beta_1\ sexo_i + \beta_2\ edad_i + \beta_3\ adherencia_{ij}
	+ \beta_4\ mes_j + \beta_5\ mes_j\ adherencia_{ij} + b_{1i}\ mes_j + \varepsilon_{ij}
\]

%%%%%%%%%%%%%%%%%%
% HECHO CON R
%%%%%%%%%%%%%%%%%%

\begin{table}[H]
	\centering
	\caption{Modelo con pendiente aleatoria}
	\label{modelo_time}
	\begin{tabular}{*{5}{|c}|}
		\hline
		\multicolumn{3}{|c}{Log-Likelihood} & \multicolumn{2}{|c|}{-15572,14} \\
		\multicolumn{3}{|c}{Var($mes_j$)} & \multicolumn{2}{|c|}{2,24} \\
		\hline
		Covariable 				 & Coef.   & Std. Err. & z      & $P<|z|$  \\
		\hline
		$intercepto$             & 122,914 & 1,946     & 63,162 & $<0.001$ \\
		$sexo_i$                 & 3,8     & 0,566     & 6,713  & $<0.001$ \\
		$edad_i$                 & 0,156   & 0,029     & 5,361  & $<0.001$ \\
		$adherencia_{ij}$        & -1,539  & 1,091     & -1,41  & $0.158$  \\
		$mes_j$                  & 0,376   & 0,255     & 1,474  & $0.141$  \\
		$mes_j\ adherencia_{ij}$ & -0,798  & 0,276     & -2,886 & $0.004$  \\
		\hline
	\end{tabular}
\end{table}

Y se plantea el siguiente test de hipótesis:

$ \left. H_0 \right) D_{00}\ =\ D_{01}\ =\ 0\ \quad\ \left. H_1\ \right) al\ menos\ uno\ distinto\ de\ cero $

Siendo $G^2 = (-2)*LogLikelihood$, MC el modelo completo y MR el modelo
reducido obtenemos:

$ G^2(MR) - G^2(MC) = 31144.28 - 30787.86 = 356.52 $

Al comparar este valor con una $\chi_{50:50;1;0.05} = 5.14$ resulta mayor, por
lo tanto se rechaza la hipótesis nula y se considera que se debe incluir la
ordenada al origen aleatoria.

Al repetir los mismos pasos para un modelo solo con ordenada aleatoria se
obtienen los resultados presentados en la tabla \ref{modelo_intercept}:

\[
	Y_{ij} = \beta_0 + b_{0i} + \beta_1\ sexo_i + \beta_2\ edad_i + \beta_3\ adherencia_{ij}
	+ \beta_4\ mes_j + \beta_5\ mes_j\ adherencia_{ij} + \varepsilon_{ij}
\]

\begin{table}[H]
	\centering
	\caption{Modelo con ordenada aleatoria}
	\label{modelo_intercept}
	\begin{tabular}{*{5}{|c}|}
		\hline
		\multicolumn{3}{|c}{Log-Likelihood} & \multicolumn{2}{|c|}{-15419,49} \\
		\multicolumn{3}{|c}{Var($intercepto$)} & \multicolumn{2}{|c|}{61,168} \\
		\hline
		Covariable 				 & Coef.   & Std. Err. & z      & $P<|z|$  \\
		\hline
		$intercepto$             & 122.102 & 2.482     & 49.204 & $<0.001$ \\
		$sexo_i$                 & 3.771   & 0.752     & 5.016  & $<0.001$ \\
		$edad_i$                 & 0.165   & 0.039     & 4.261  & $<0.001$ \\
		$adherencia_{ij}$        & -1.129  & 1.086     & -1.039 & $0.299$  \\
		$mes_j$                  & 0.430   & 0.224     & 1.924  & $0.054$  \\
		$mes_j\ adherencia_{ij}$ & -0.866  & 0.250     & -3.467 & $0.001$  \\
		\hline
	\end{tabular}
\end{table}

Se plantea el test de hipótesis:

$ \left. H_0 \right) D_{11}\ =\ D_{01}\ =\ 0\ \quad\ \left. H_1\ \right) al\ menos\ uno\ distinto\ de\ cero $

Siendo $G^2 = (-2)*Log-Likelihood$, MC el modelo completo y MR el modelo
reducido obtenemos:

$ G^2(MR) - G^2(MC) = 30838.98 - 30787.86 = 51.12 $

Al comparar este valor con una $\chi_{50:50;1;0.05} = 5.14$ vemos es que mayor,
por lo tanto se rechaza la hipótesis nula y se considera que se debe incluir la
pendiente aleatoria.

En conclusión, ambos efectos aleatorios son añadidos al modelo.

\newpage
\nocite{*}
\renewcommand{\refname}{Bibliografía}
\bibliography{Bibliografia}

\end{document}