\documentclass[12pt]{article}
\usepackage[a4paper, left=1in, right=1in, top=1in, bottom=1in]{geometry}
                                % for page size and margin settings
\usepackage{graphicx}		    % for insert images
\usepackage[spanish]{babel} 	% for spanish titles
\usepackage[a4paper, left=1in, right=1in, top=1in, bottom=1in]{geometry} 
								% for page size and margin settings
\usepackage{mathtools}          % for greek math symbol formatting
\usepackage{enumitem}           % for control of 'enumerate' numbering
\usepackage{listings}           % for control of 'itemize' spacing
\usepackage{indentfirst}		% package to make first paragraph always indented
\usepackage{hyperref}           % page numbers and '\ref's become clickable
\usepackage{bm}					% for bold maths
\usepackage{hyperref}           % page numbers and '\ref's become clickable
\usepackage{setspace}			% for setting interline spacing
\renewcommand{\baselinestretch}{1.5}
\bibliographystyle{ieeetr}

% TITLE VARIABLES 

\def\thesistitle{Incorporación de covariables que varían en el tiempo a un modelo mixto}
\def\thesisauthorfirst{\textbf{Esteban Cometto}}
\def\thesissupervisorfirst{Noelia Castellana}
\def\thesissupervisorsecond{Cecilia Rapelli}
\def\thesisdate{\today}

%% OTHER USEFUL VARIABLES 

\def\npatients{560}
\def\covname{\emph{adherencia}}
\def\cvt{covariable que varía en el tiempo}
\def\xseqj{$X_{i1}, ..., X_{ij}$}
\def\xseqn{$X_{i1}, ..., X_{in_i}$}
\def\xseqjminus{$X_{i1}, ..., X_{ij-1}$}
\def\yseqj{$Y_{i1}, ..., Y_{ij}$}
\def\yseqn{$Y_{i1}, ..., Y_{in_i}$}
\def\yseqjminus{$Y_{i1}, ..., Y_{ij-1}$}

%% FOR PDF METADATA
\title{\thesistitle}
\author{\thesisauthorfirst\space\thesisauthorsecond}
\date{\thesisdate}

\begin{document}

\begin{titlepage}
    \newcommand{\HRule}{\rule{\linewidth}{0.5mm}}
	\center
	\textsc{\Large Universidad Nacional de Rosario}\\[.7cm]
	\includegraphics[width=25mm]{img/fceye-unr.png}\\[.5cm]
	\textsc{Facultad de Ciencias Económicas y Estadística}\\[0.5cm]
	\textsc{Anteproyecto de Tesina}
	
	\HRule \\[0.4cm]
	{ \huge \bfseries \thesistitle}\\[0.1cm]
	\HRule \\[.5cm]
	
	\begin{minipage}{0.6\textwidth}
	\large
	\emph{Autor:}	\thesisauthorfirst
	\end{minipage}
	\\[.6cm]
	\begin{minipage}{0.6\textwidth}
	\emph{Directora:} 	\thesissupervisorfirst \\[.2cm]
	\emph{Codirectora:} 	\thesissupervisorsecond
	\end{minipage}
	\\[4cm]
	\vfill
	{\large \thesisdate}\\
	\clearpage
\end{titlepage}

\newpage
\tableofcontents

\newpage
\section{Introducción}

Los estudios longitudinales están conformados por datos obtenidos midiendo repetidamente una variable respuesta a
la misma unidad. En este tipo de estudios es también frecuente contar con variables explicativas que se desean incorporar
al análisis. Estas covariables pueden ser fijas a lo largo de todo el período (por ejemplo el sexo biológico de una persona)
o bien puedan variar a lo largo del seguimiento (por ejemplo el valor de colesterol).

Los modelos lineales mixtos permiten
analizar este tipo de datos, modelando, por un lado, la evolución de la respuesta promedio en función del tiempo y las
covariables, mediante efectos fijos, y, por otro lado, la variación entre las respuestas repetidas dentro y entre sujetos
por medio del error y los efectos aleatorios, respectivamente. Las covariables que varían en el tiempo pueden
utilizarse para comparar poblaciones, para describir tendencias en el tiempo, y también para describir relaciones
dinámicas con la variable respuesta. La relación entre la covariable que varía en el tiempo y la variable respuesta
puede estar confundida por valores anteriores y/o posteriores de la covariable y en consecuencia esto puede conducir
a inferencias engañosas sobre los parámetros del modelo. Esta tesina realiza una introducción a la problemática de
incorporar covariables que varían con el tiempo en modelos para datos longitudinales, presentando diferentes definiciones
de las mismas y enfoques metodológicos aplicando los conceptos en un estudio de hipertensión arterial. 


\newpage
\section{Objetivos}

\subsection{Objetivo Principal}

Presentar diferentes propuestas metodológicas respecto a la incorporación de covariables que varían con el tiempo en
modelos mixtos para datos longitudinales.

\subsection{Objetivos Específicos}

\begin{itemize}
	\item Definir los tipos de covariables existentes.
	\item Evaluar propuestas de incorporación de covariables que varían en el tiempo en los modelos mixtos.
	\item Aplicar los conceptos vistos en un estudio sobre la tendencia de la presión arterial en el tiempo para pacientes
		  que siguen cierto tratamiento.
\end{itemize}

\newpage
\section{Metodología}

En un modelo mixto, cada unidad tiene una trayectoria individual caracterizada por parámetros individuales.
La respuesta media es modelada como una combinación de características poblacionales que son comunes a todos los
individuos (efectos fijos) y efectos específicos de la unidad que son únicos de ella (efectos aleatorios),
considerando las dos fuentes de variación (intra y entre) presentes en los datos longitudinales.

Este modelo supone implícitamente que la media condicional de la respuesta j-ésima, dado \xseqn{}, depende solo de $X_{ij}$.
\begin{equation}
\label{esperanza}
	E(Y_{ij}|\bm{X}_i) = E(Y_{ij}|X_{i1}, ..., X_{in_i}) = E(Y_{ij}|X_{ij})
\end{equation}

Con las covariables fijas en el tiempo, esta suposición se mantiene necesariamente ya que $X_{ij} = X_{ik}$ para todas las ocasiones
$k \neq j$. Con covariables variables en el tiempo (CVT) que se fijan por diseño del estudio (por ejemplo, indicador
de grupo de tratamiento en una prueba cruzada), la suposición también se cumple ya que los valores de las covariables
en cualquier ocasión se determinan a priori por diseño del estudio y de manera completamente no relacionado con la
respuesta longitudinal. Sin embargo, cuando una covariable es variable en el tiempo, puede que no necesariamente se mantenga.

En general, cuando (\ref{esperanza}) no se cumple, los valores precedentes y/o posteriores de la CVT confunden la
relación entre $Y_{ij}$ y $X_{ij}$, esto puede llevar a estimaciones sesgadas de los parámetros del modelo.

Se dice que una \cvt{} es exógena cuando los valores actuales y anteriores de la respuesta en la ocasión
$j$ (\yseqj{}), dados los valores actuales y precedentes de la CVT (\xseqj{}), no predicen el valor
posterior de $X_{ij+1}$. De lo contrario, se dice que la CVT es endógena.

Cuando las covariables son variables en el tiempo, los parámetros de regresión no
necesariamente tienen la interpretación causal implícita incluso cuando (\ref{esperanza}) se cumple. A los parámetros de
regresión se les puede dar una interpretación causal solo cuando se puede asumir además que las CVT son exógenas con
respecto a la variable de respuesta.

Existen diferentes formas de incorporar las CVT al modelo según los objetivos de la investigación y el tipo de covariable
(endógena / exógena) ya sea realizando transformaciones de interés o bien considerando la covariable sin transformar.

Puede ser de interés resumir la información de la CVT para cada individuo en un único valor e incorporarla como una covariable
fija en el tiempo, sin embargo de esta forma se pierde la información longitudinal de la covariable y la interpretación de los
parámetros del modelo resulta confusa.

Otra forma de incorporar la CVT es dividiendo su efecto en dos componentes: efecto entre-individuos y efecto intra-individuos e
introducirla como dos covariables diferentes.

También pueden realizarse transformaciones a la CVT:  resumiendo los valores observados hasta el momento $t$
(por ejemplo el promedio), considerando mediciones rezagadas, etc e introducir al modelo las covariables transformadas.
Si se desea incorporar la CVT sin ninguna transformación es necesario evaluar si la covariable es exógena. En caso contrario
(covariable endógena) se requiere aplicar modelos más sofisticados.

\newpage
\section{Materiales}

Los datos que se utilizarán provienen de un programa de atención y control de pacientes hipertensos iniciado en el año 2014
en Rosario que realiza un seguimiento exhaustivo de \npatients{} pacientes. Este programa contempla: efectores no médicos
supervisados, tratamiento farmacológico genérico para la hipertensión y utilización de un algoritmo terapéutico sistematizado.
En cada visita se registran tanto características de los pacientes, del tratamiento y de los valores de la tensión arterial.
En particular, se desea evaluar si la variable ``adherencia al tratamiento farmacológico'' influye en los valores de la tensión arterial
sistólica a lo largo del seguimiento. Dicha variable se evaluó a través de un cuestionario validado que otorga como resultado
una respuesta dicotómica: adhiere - no adhiere. Este cuestionario se realizó en cada visita. Como la adherencia al tratamiento
farmacológico es una CVT, se evaluaran diferentes enfoques para incluirla en un modelo mixto que pueda explicar el cambio
en la tensión arterial sistólica media a lo largo del tiempo.

Un aspecto a tener en cuenta en este trabajo es que, si bien contamos con mucha otra información para obtener modelos
que describan de mejor manera el comportamiento de la tensión arterial sistólica, nos centraremos en modelos más simples con
respecto a las covariables fijas con el fin de no perder de vista la relación entre la variable respuesta y la CVT.

Para el desarrollo del análisis de los datos se utilizará el lenguaje Python.

\newpage
\section{Cronograma}

\subsection*{Octubre 2020 - Julio 2021}
\begin{itemize}
	\item Revisión bibliográfica sobre covariables que varían en el tiempo para modelos longitudinales
\end{itemize}

\subsection*{Agosto 2021 - Octubre 2021}
\begin{itemize}
	\item Procesamiento y análisis de la información 
\end{itemize}

\subsection*{Noviembre 2021 - Abril 2022}
\begin{itemize}
	\item Elaboración de informe
\end{itemize}

\subsection*{Mayo 2022 - Junio 2022}
\begin{itemize}
	\item Revision de informe
\end{itemize}

\subsection*{Julio 2022}
\begin{itemize}
	\item Fecha tentativa de presentación
\end{itemize}

\newpage
\nocite{*}
\renewcommand{\refname}{Bibliografía}
\bibliography{Bibliografia}

\end{document}