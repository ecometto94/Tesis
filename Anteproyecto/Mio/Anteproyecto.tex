\documentclass[12pt]{article}
\usepackage[a4paper, left=1in, right=1in, top=1in, bottom=1in]{geometry}
                                % for page size and margin settings
\usepackage{graphicx}		    % for insert images
\usepackage[spanish]{babel} 	% for spanish titles
\usepackage[a4paper, left=1in, right=1in, top=1in, bottom=1in]{geometry} 
								% for page size and margin settings
\usepackage{mathtools}          % for greek math symbol formatting
\usepackage{enumitem}           % for control of 'enumerate' numbering
\usepackage{listings}           % for control of 'itemize' spacing
\usepackage{indentfirst}		% package to make first paragraph always indented
\usepackage{hyperref}           % page numbers and '\ref's become clickable
\usepackage{bm}					% for bold maths
\usepackage{hyperref}           % page numbers and '\ref's become clickable
\usepackage{setspace}			% for setting interline spacing
\renewcommand{\baselinestretch}{1.5}
\bibliographystyle{ieeetr}

% TITLE VARIABLES 

\def\thesistitle{Modelos longitudinales con covariables que varían en el tiempo}
\def\thesisauthorfirst{\textbf{Esteban Cometto}}
\def\thesissupervisorfirst{Noelia Castellana}
\def\thesissupervisorsecond{Cecilia Rapelli}
\def\thesisdate{\today}

%% OTHER USEFUL VARIABLES 

\def\npatients{560}
\def\fullcovname{adherencia al tratamiento farmacológico}
\def\covname{\emph{adherencia}}
\def\cvt{covariable que varía en el tiempo}
\def\xseqj{$X_{i1}, ..., X_{ij}$}
\def\xseqn{$X_{i1}, ..., X_{in_i}$}
\def\xseqjminus{$X_{i1}, ..., X_{ij-1}$}
\def\yseqj{$Y_{i1}, ..., Y_{ij}$}
\def\yseqn{$Y_{i1}, ..., Y_{in_i}$}
\def\yseqjminus{$Y_{i1}, ..., Y_{ij-1}$}

%% FOR PDF METADATA
\title{\thesistitle}
\author{\thesisauthorfirst\space\thesisauthorsecond}
\date{\thesisdate}

\begin{document}

\begin{titlepage}
    \newcommand{\HRule}{\rule{\linewidth}{0.5mm}}
	\center
	\textsc{\Large Universidad Nacional de Rosario}\\[.7cm]
	\includegraphics[width=25mm]{img/fceye-unr.png}\\[.5cm]
	\textsc{Facultad de Ciencias Económicas y Estadística}\\[0.5cm]
	\textsc{Anteproyecto de Tesina}
	
	\HRule \\[0.4cm]
	{ \huge \bfseries \thesistitle}\\[0.1cm]
	\HRule \\[.5cm]
	
	\begin{minipage}{0.6\textwidth}
	\large
	\emph{Autor:}	\thesisauthorfirst
	\end{minipage}
	\\[.6cm]
	\begin{minipage}{0.6\textwidth}
	\emph{Directora:} 	\thesissupervisorfirst \\[.2cm]
	\emph{Codirectora:} 	\thesissupervisorsecond
	\end{minipage}
	\\[4cm]
	\vfill
	{\large \thesisdate}\\
	\clearpage
\end{titlepage}

\newpage
\tableofcontents

\newpage
\section{Introducción}

Un estudio longitudinal se realiza cuando el objetivo de la investigación es caracterizar
el cambio de la variable en estudio a través del tiempo y los factores que influyen en ese
cambio. Éstos consisten en medir repetidamente la variable de interés a cada uno de los
individuos o unidades a través del tiempo, bajo diferentes condiciones experimentales permitiendo
caracterizar  el  cambio  en  la respuesta y los factores que influencian dicho cambio.

Los modelos lineales mixtos permiten analizar este tipo de datos, modelando, por un lado, la evolución de la respuesta promedio en
función del tiempo, mediante efectos fijos, y, por otro lado, la variación entre las respuestas repetidas dentro y entre
sujetos por medio del error y los efectos aleatorios, respectivamente. En este tipo de estudios es también frecuente contar
con variables explicativas que se desean incorporar al análisis. Estas variables pueden ser fijas a lo largo de todo el
período o bien pueden variar a lo largo del seguimiento. La introducción de una \cvt{} (CVT) al
modelo produce un importante desafío conceptual. Se consideran algunos aspectos de la interpretación de este tipo de
covariables, presentando diferentes definiciones y enfoques para incorporarlas en los modelos mixtos.

\newpage
\section{Objetivos}

\subsection{Objetivo Principal}

Profundizar en el estudio de propuestas metodológicas para utilizar la información obtenida de la \cvt{} dentro de un
modelo mixto.

\subsection{Objetivos Específicos}

\begin{itemize}
	\item Específicar distintos tipos de CVT
	\item Transformaciones a realizar sobre la CVT antes de incluirla al modelo, incluyendo conversión a covariable fija
	\item Consideraciones sobre interpretación de los parámetros sobre las CVT
	\item Indagar sobre feedback entre la CVT y la variable respuesta
\end{itemize}

\newpage
\section{Metodología}

En un modelo mixto, cada unidad tiene una trayectoria individual caracterizada por parámetros y un subconjunto de esos
parámetros ahora se consideran aleatorios. La respuesta media es modelada como una combinación de características
poblacionales que son comunes a todos los individuos (efectos fijos) y efectos específicos de la unidad que son únicos de
ella (efectos aleatorios).

Se consideran las dos fuentes de variación (intra y entre) presentes en los datos longitudinales. Entonces, este modelo va
a ser similar al modelo lineal general con respecto a la parte media del mismo, pero se va a diferenciar en cuanto a la
estructura de covariancia.

Lo que a menudo se pasa por alto es la suposición implícita de que la media
condicional de la respuesta j-ésima, dado \xseqn{}, depende solo de $X_{ij}$.
\begin{equation}
\label{esperanza}
	E(Y_{ij}|\bm{X}_i) = E(Y_{ij}|X_{i1}, ..., X_{in_i}) = E(Y_{ij}|X_{ij})
\end{equation}

Con las covariables fijas, esta suposición se mantiene necesariamente desde $X_{ij} = X_{ik}$ para todas las ocasiones
$k \neq j$. Además, con covariables variables en el tiempo que se fijan por diseño del estudio (por ejemplo, indicador
de grupo de tratamiento en una prueba cruzada), la suposición también se cumple ya que los valores de las covariables
en cualquier ocasión se determinan a priori por diseño del estudio y de manera completamente no relacionado con la
respuesta longitudinal. Sin embargo, cuando una covariable es variable en el tiempo y estocástica, puede que no
necesariamente se mantenga.

En general, cuando (\ref{esperanza}) no se cumple, los valores precedentes y/o posteriores de la CVT confunden la
relación entre $Y_{ij}$ y $X_{ij}$, esto puede llevar a estimaciones sesgadas de los parámetros del modelo.

Se dice que una \cvt{} es exógena cuando los valores actuales y anteriores de la respuesta en la ocasión
$j$ (\yseqj{}), dados los valores actuales y precedentes de la CVT (\xseqj{}), no predice el valor
posterior de $X_{ij+1}$. De lo contrario, se dice que la CVT es endógena.

Cuando las covariables son variables en el tiempo, los parámetros de regresión no
necesariamente tienen la interpretación causal implícita incluso cuando (\ref{esperanza}) se cumple. A los parámetros de
regresión se les puede dar una interpretación causal solo cuando se puede asumir además que las CVT son exógenas con
respecto a la variable de respuesta.

\newpage
\section{Materiales}

Un programa de atención y control de pacientes hipertensos iniciado en el año 2014 en Rosario realiza un seguimiento
exhaustivo de \npatients{} pacientes. Este programa contempla: efectores no médicos supervisados, tratamiento farmacológico
genérico para la hipertensión y utilización de un algoritmo terapéutico sistematizado. En cada visita se registran tanto
características de los pacientes, del tratamiento y de los valores de la tensión arterial. En particular, se desea evaluar
si la adherencia al tratamiento farmacológico influye en los valores de la tensión arterial sistólica a lo largo del
seguimiento. Como la variable “\fullcovname{}” es una CVT estocástica se evaluaran diferentes enfoques para incluirla en un
modelo longitudinal que pueda explicar el cambio en la tensión arterial sistólica media a lo largo del tiempo.

Un aspecto a tener en cuenta en este trabajo es que, si bien contamos con mucha otra información para
obtener modelos que describan de mejor manera el comportamiento de la TAS, nos centraremos en modelos más simples con
respecto a las covariables fijas con el fin de no perder de vista la relación entre la variable respuesta y la CVT.

\newpage
\section{Cronograma}

\subsection*{Septiembre 2021 - Octubre 2021}
\begin{itemize}
	\item Revisión bibliográfica sobre covariables que varían en el tiempo para modelos longitudinales
\end{itemize}

\subsection*{Noviembre 2021 - Diciembre 2021}
\begin{itemize}
	\item Procesamiento y análisis de la información 
\end{itemize}

\subsection*{Enero 2021 - Febrero 2022}
\begin{itemize}
	\item Elaboración de informe
\end{itemize}

\subsection*{Marzo 2022}
\begin{itemize}
	\item Revision de informe
\end{itemize}

\subsection*{Abril 2022}
\begin{itemize}
	\item Fecha tentativa de presentación
\end{itemize}

\newpage
\nocite{*}
\renewcommand{\refname}{Bibliografía}
\bibliography{Bibliografia}

\end{document}